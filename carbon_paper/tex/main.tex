% mnras_template.tex 
%
% LaTeX template for creating an MNRAS paper
%
% v3.0 released 14 May 2015
% (version numbers match those of mnras.cls)
%
% Copyright (C) Royal Astronomical Society 2015
% Authors:
% Keith T. Smith (Royal Astronomical Society)

% Change log
%
% v3.0 May 2015
%    Renamed to match the new package name
%    Version number matches mnras.cls
%    A few minor tweaks to wording
% v1.0 September 2013
%    Beta testing only - never publicly released
%    First version: a simple (ish) template for creating an MNRAS paper

%%%%%%%%%%%%%%%%%%%%%%%%%%%%%%%%%%%%%%%%%%%%%%%%%%
% Basic setup. Most papers should leave these options alone.
\documentclass[fleqn,
%referee, % this line changes to double-spaced 1 column
usenatbib]{mnras}


% MNRAS is set in Times font. If you don't have this installed (most LaTeX
% installations will be fine) or prefer the old Computer Modern fonts, comment
% out the following line
\usepackage{newtxtext,newtxmath}
\usepackage{anyfontsize}
% Depending on your LaTeX fonts installation, you might get better results with one of these:
%\usepackage{mathptmx}
%\usepackage{txfonts}

% Use vector fonts, so it zooms properly in on-screen viewing software
% Don't change these lines unless you know what you are doing
\usepackage[T1]{fontenc}

% Allow "Thomas van Noord" and "Simon de Laguarde" and alike to be sorted by "N" and "L" etc. in the bibliography.
% Write the name in the bibliography as "\VAN{Noord}{Van}{van} Noord, Thomas"


%%%%% AUTHORS - PLACE YOUR OWN PACKAGES HERE %%%%%

% Only include extra packages if you really need them. Common packages are:
\usepackage{graphicx}	% Including figure files
\usepackage{amsmath}	% Advanced maths commands
% \usepackage{amssymb}	% Extra maths symbols
\usepackage{hyperref}
\usepackage[normalem]{ulem}
\usepackage[dvipsnames]{xcolor}
% \usepackage{enumitem}

\usepackage{tikz}


\graphicspath{{figures/}} 
% \linespread{1.8}




%%%%%%%%%%%%%%%%%%%%%%%%%%%%%%%%%%%%%%%%%%%%%%%%%%

%%%%% AUTHORS - PLACE YOUR OWN COMMANDS HERE %%%%%

% Please keep new commands to a minimum, and use \newcommand not \def to avoid
% overwriting existing commands. Example:
%\newcommand{\pcm}{\,cm$^{-2}$}	% per cm-squared


% citations
\defcitealias{james+21}{J21}
\newcommand{\JJ}{\citetalias{james+21}}
\newcommand{\VICE}{\textsc{vice}}


\newcommand{\fruity}{\texttt{\hyperlink{fruity}{Fruity}}}
\newcommand{\nugrid}{\texttt{\hyperlink{nugrid}{NuGrid}}}
\newcommand{\monash}{\texttt{\hyperlink{monash}{Monash}}}
\newcommand{\aton}{\texttt{\hyperlink{aton}{ATON}}}
\newcommand{\cfactor}{1.44}
\newcommand{\nsubgiants}{14,069}



% Acronyms
\newcommand{\agb}{AGB}
\newcommand{\apogee}{APOGEE}
\newcommand{\aspcap}{\textsc{aspcap}}
\newcommand{\cc}{CCSN}
\newcommand{\gce}{GCE}
\newcommand{\ia}{SN Ia}

% internal abbreviations
\newcommand{\caah}{[C/Mg]-[Mg/H]}
\newcommand{\caafe}{[C/Mg]-[Mg/Fe]}


% other
\newcommand{\ycmg}{\ensuremath{2.7 + 32\left(Z-\Zo\right)}}
\newcommand{\fmeas}{20\%}

\makeatletter
\newcommand{\C}[1][\@nil]{
    \def\tmp{#1}%
    \ifx\tmp\@nnil%
        \ensuremath{\rm C}%
    \else%
        \ifmmode ^{#1}{\rm C}%
        \else $^{#1}$C%
        \fi%
\fi }
\makeatother

\newcommand{\Yct}{{y_{\rm C}}}
\newcommand{\Ycc}{{y_{\rm C}^{\rm CC}}}
\newcommand{\Yoc}{{y_{\rm Mg}^{\rm CC}}}
\newcommand{\Ycagb}{{y_{\rm C}^{\rm AGB}}}
\newcommand{\ycagb}{\y_{\rm C}^{\rm AGB}}
\newcommand{\aagb}{\alpha_{\rm C}^{\rm AGB}}
\newcommand{\zagb}{\zeta_{\rm C}^{\rm AGB}}
\newcommand{\zcc}{\zeta_{\rm C}^{\rm CC}}
\newcommand{\noneqagb}{\nu_{\rm C}^{\rm AGB}}
 \newcommand{\yb}{\ensuremath{\rotatebox[origin=B,y=0.5ex]{180}{y}}}
\newcommand{\y}{Y}
\newcommand{\fagb}{f_{\rm C}^{\rm AGB}}

\newcommand{\zetao}{\zeta^{(0)}}
\newcommand{\zetai}{\zeta^{(1)}}
\newcommand{\zetaii}{\zeta^{(2)}}

\newcommand{\Mo}{ {\rm M}_{\sun}}
    
\newcommand{\Zo}{ Z_{\sun}}

\newcommand{\about}[1]{${\sim} #1$}

%%% citepos command
\makeatletter
\DeclareRobustCommand\citepos
  {\begingroup
   \let\NAT@nmfmt\NAT@posfmt% ...except with a different name format
   \NAT@swafalse\let\NAT@ctype\z@\NAT@partrue
   \@ifstar{\NAT@fulltrue\NAT@citetp}{\NAT@fullfalse\NAT@citetp}}

\let\NAT@orig@nmfmt\NAT@nmfmt
\def\NAT@posfmt#1{\NAT@orig@nmfmt{#1's}}

\makeatother


% Caption macros
\newcommand{\captionline}[2][very
thick]{\tikz[baseline={([yshift=-.5ex]current bounding box.base)}]{
\draw[#2,#1, line cap=round] (0,0) -- (0.3,0); }}




%%%%%%%%%%% JWJ' editing macros %%%%%%%%%%%
\newcommand{\strike}[1]{{\color{ForestGreen} \sout{#1}}}
\newcommand{\add}[1]{{\color{ForestGreen} #1}}
\newcommand{\note}[1]{{\color{ForestGreen} \textit{ \small (JWJ: #1)}}}
% \LetLtxMacro\origcitep\citep
% \LetLtxMacro\origcitet\citet
% \renewcommand{\citep}[2][]{%
%     \mbox{\origcitep[#1][]{#2}}
% }
% \renewcommand{\citet}[2][]{%
%     \mbox{\origcitet[#1][]{#2}}
% }

\newcommand{\dbstrike}[1]{{\color{Thistle} \sout{#1} }}
\newcommand{\dbadd}[1]{{\color{Thistle} #1}}
\newcommand{\dbnote}[1]{ {\color{Thistle} \textit{\small (DB: #1)}} }


% \LetLtxMacro\origcite\cite

% This is my command. I want to strike out, 
% \newcommand{\strikeit}[1]{%
%   \ifcorrectingmode
%   \mbox{\sout{#1}}%
% %  \makebox[\textwidth][s]{\sout{#1}}%
%   \fi
% }

% \renewcommand{\cite}[2][]{%
%   \ifcorrectingmode
%   \mbox{\origcite[#1]{#2}}%
%   \else
%   \origcite[#1]{#2}%
%   \fi
% }




%%%%%%%%%%%%%%%%%%%%%%%%%%%%%%%%%%%%%%%%%%%%%%%%%%

%%%%%%%%%%%%%%%%%%% TITLE PAGE %%%%%%%%%%%%%%%%%%%

% Title of the paper, and the short title which is used in the headers.
% Keep the title short and informative.
\title[The Origin and Galactic Evolution of Carbon]{The Galactic Chemical Evolution of Carbon: Implications for Stellar Nucleosynthesis }

% The list of authors, and the short list which is used in the headers.
% If you need two or more lines of authors, add an extra line using \newauthor
\author[D. A. Boyea et. al.]{%
Daniel A. Boyea,$^{1, 2, 3}$\thanks{%
Contact e-mail:~\href{mailto:boyea.2@osu.edu}{boyea.2@osu.edu}}
James W. Johnson,$^{1, 2, 4}$
Third Author$^{2,3}$
and Others$^{1,3}$
\\
% List of institutions
$^{1}$Department of Astronomy, The Ohio State University, 140 W. 18th Ave., Columbus, OH, 43210, USA
\\
$^{2}$Center for Cosmology \& Astroparticle Physics (CCAPP), The Ohio State University, 191 W. Woodruff Ave., Columbus, OH, 43210, USA
\\
$^{3}$Department of Physics \& Astronomy, University of Victoria, Victoria, BC V8P 5C2, Canada
\\
$^{4}$The Observatories of the Carnegie Institution for Science, 813 Santa Barbara St., Pasadena, CA, 91101, USA
}

% These dates will be filled out by the publisher
\date{Accepted XXX. Received YYY; in original form ZZZ}

% Enter the current year, for the copyright statements etc.
\pubyear{2024}

% Don't change these lines
\begin{document}
\label{firstpage}
\pagerange{\pageref{firstpage}--\pageref{lastpage}}
\maketitle



% Abstract of the paper
\begin{abstract}
% context
The astrophysical origin of carbon remains largely uncertain---different studies propose that either asymptotic giant branch (AGB) stars or core collapse supernovae (CCSNe) may be the dominant process.
% 
We aim to constrain the stellar yields of C through multi-zone Galactic chemical evolution models by comparing predictions with \apogee\ subgiants abundances.
% 
We find that \caafe\ trends, when restricted in metallicity, are an empirical estimate of the delayed C sources, enabling us to estimate that \agb\ stars and \cc\ produce about 20\% and 80\% of C, respectively.  
The \caah\ trend instead represents the equilibrium abundances of C and Mg. 
We use the \caah\ trend to estimate total C/Mg yield ratio, determining that  $\Ycc/\Yoc = EQUATION$.
% misc
Our models are relatively independent of uniform scaling of yields and outflows, and alternate star formation histories. 
However, the stars which contribute to \agb\ C production and the \ia\ delay time distribution of Fe contribute uncertainties to our conclusions. 
% gas phase
While reliable gas-phase and low-metallicity measurements of C are challenging, we find that our model and a single-zone model with our recommended yields replicate the broad trends of \caah{} across different environments and metallicities. 

\end{abstract}

\begin{keywords}
galaxies: abundances -- galaxies: evolution -- galaxies: star formation -- galaxies: stellar content -- methods: numerical
\end{keywords}



%%%%%%%%%%%%%%%%%%%%%%%%%%%%%%%%%%%%%%%%%%%%%%%%%%

%%%%%%%%%%%%%%%%% BODY OF PAPER %%%%%%%%%%%%%%%%%%

\section{Introduction}


\begin{figure*}
    \centering
    \includegraphics{subgiants.pdf}
    \caption{The [C/Mg] ratio against [Mg/H] (left) and [Mg/Fe] (right) for the \citet{jack}~sample of \apogee{} subgiants. On the left, we plot high and low-$\alpha$ stars in blue and orange, using the separation defined in Eq.~\ref{eq:high_alpha}.  On the right, we colour-code stars according to their [Mg/H] abundance.} \label{fig:subgiants}
\end{figure*}



Despite being the most abundant metal in the Universe~\citep[e.g.][]{magg+22}, the nucleosynthetic origin of C is poorly understood. 
There is broad agreement that both high mass and asymptotic giant branch (AGB) stars produce C \citep{jennifer19}, but their relative yields are disputed.
From a Galactic chemical evolution (GCE) perspective, some authors have argued that high-mass stars should dominate~\citep[e.g.][]{prantzos+94, HEK00, romano+20, franchini+20, gustafsson22}, and others have argued that AGB stars should contribute at least half of the C in the universe~\citep[e.g.][]{tinsley79, chiappini+03, mattsson10, KKU11, rybizki+17, KKL20}. 
In this paper, we aim to constrain C production using subgiant stars and multi-zone GCE models \dbstrike{incorporating stellar radial migration}.

% Formed in the cores of stars during He fusion, C is the first \dbadd{primary? element} after He. Additionally, C is one of the only light elements formed in low-mass stars \citep[e.g.,][]{KL14}. \dbnote{I don't like how many e.g.'s there are, and is this the right citation here?}
%\dbadd{
% Because the effects of first dredge up are mass-dependent, C and N abundances are frequently used as age indicators for RGB stars \citep{MG15, martig16, hasselquist19, vincenzo+21}. 
% }
% \dbnote{C abundnces for stellar structure (opacities) and star formation (line cooling)}
% \dbadd{In addition, stellar parameters such as convection and mass loss are calibrated to match observed stars and cannot be predicted from first principles.} \dbnote{need to check if this means SE models are calibrated to match yields or just stellar parameters.}


Predictions of C yields from stellar evolution and supernova (SN) models are riddled with uncertainties (see, e.g., the reviews by \citealt{romano+10, KL14}).
Some \dbadd{substantial uncertanties} include mass loss rates \citep{sukhbold+16, beasor+2020}, rotational mixing \citep{frischknecht+16, LC18}, nuclear reaction rates \dbadd{CITE}, and convection \citep{chieffi2001, ventura+13}.
Recent work also suggests that small changes in a stars's initial mass of 
 $\sim$$0.01 M_\odot$  have significant effects on the evolution and explosion outcomes of massive stars \citep{bruenn+2023, vartanyan_burrows2023}.
Furthermore, many \add{massive} stars are found in binary systems \dbadd{\url{https://ui.adsabs.harvard.edu/abs/2012Sci...337..444S/abstract} ??}, the effects of which are almost entirely
unexplored in stellar nucleosynthesis models \citet{farmer+21}.
Accurate predictions of metal yields from stellar evolution models remain elusive.

Given the challenge of {\it a priori} yield predictions, recent work aims to empirically characterize elemental production \dbnote{does it make sense to frame this as only recent work? I just don't know but I feel like there are very early examples which at least use observations as a guide for yields, like maybe even B2FH. or garnett1997}
\citet{weinberg+19, weinberg+22} analyzed the abundance ratio trends of high- and low-alpha sequence stars in APOGEE~\citep{apogee17} to infer relative yields of prompt and delayed enrichment sources and the metallicity dependence thereof.
\citet{emily+19, emily+22, emily+24} applied the same methodology to GALAH~\citep{DeSilva2015, Martell2017}.
\citet{rodriguez+21, rodriguez+23} measured the mean Fe yield of core-collapse supernovae (CCSNe) using the radioactive tails of their lightcurves.
Their measurements, provide one of few empirical anchors on the absolute scale of yields \citep{david_fe}.
\dbadd{Additionally the series of papers including \citet{eitner+2020} includes some relevant work.}


 % Folowing many previous works  (e.g. \citealt{DTS78, BF06, prantzos+18, berg+19}; see also the recent reviews by \citealt{romano22} and \citealt{RM21}.) this paper focuses on the evolution of C in GCE models.
\citet{james+24} recently argued that the enrichment history of the Galactic disk can be described as an equilibrium phenomenon.
In equilibrium GCE, trends in abundance ratios closely trace yield ratios (see their section 5.2.2).
\citet{james+23} had previously shown that the correlation between gas-phase N and O abundances (see their Fig. 1) could be described as a consequence of the metallicity dependence of the relative N and O yields.
Under their equilibrium scenario, this argument should extend to any element whose enrichment by stellar populations does not require long timescales ($\gtrsim$few Gyr).
We show in \add{REF 3.1} that the characteristic delay times of C production should be $\sim$ few hundred Myr, within this limit. 
\dbnote{This is complicated by the difference between high alpha and low alpha, and 1-2Msun stars do produce carbon...}


We use a sample of subgiant stars from APOGEE \citep{apogee17} as our primary observational constraint.
APOGEE Subgiants provide a large, homogeneous sample of well-measured C abundances unaffected by stellar evolution.
\dbadd{Subgiants' photospheric C abundance closely matches their birth composition.}
\dbstrike{Subgiants have the advantage that their photospheric abundances of C directly reflect their birth C abundances much more closely than other spectral types.} \citep{gilroy89, korn+07, lind+08, souto+18, souto19}.
In other evolutionary phases, the measured abundances are known to be affected by internal processes (see discussion in section 2 below).
Our sample should therefore be unaffected by uncertainties with inferring their birth abundances.


% \dbnote{not sure the observations below need to be  too detailed here? Maybe just state that C/O appears to have a universal shape across environments and direct to that section since our main focus is on APOGEE? and disk chemistry?  }
% \dbstrike{
% From observations of nearby stars and (extragalactic) gas, we know that [C/O] against [O/H] traces a banana shape (see Fig.~\ref{fig:gas_phase} and section~\ref{sec:gas}).
% At the very lowest metallicities ($[{\rm O/H}]\lesssim-2$),%
% %
% observations indicate that [C/Mg] declines with metallicity.
% From $-2 \lesssim {\rm [O/H]}\lesssim -1$, [C/Mg] is roughly constant with metallicity. 
% At higher metallicities ([O/H]~$\gtrsim -1$), [C/Mg] increases with metallicity.
% }
% \dbstrike{We will show in section~\ref{sec:results} below that these correlations in the bulk population abundances alone are quite constraining for the origin of C.}


In Section~\ref{sec:data_selection} we describe the selection for our subgiant observational sample.
In Section~\ref{sec:nucleosynthesis}, we discusses theoretical estimates of \agb\ and \cc\ C yields and present the yield prescriptions we use for our models.
In Section~\ref{sec:vice}, we describe the remaining details of our GCE models.
In Section~\ref{sec:results}, we present our model predictions and discuss their observationally distinguishable predictions. 
In Section~\ref{sec:gas}, we compare our model to observations of halo stars and extragalactic HII regions from the literature. 
We summarize our conclusins in section~\ref{sec:conclusions}.





% \footnotetext{\strike{By metallicity, we mean the (mass) fraction of any element which is not H or He, denoted by $Z$. For the sun, we take $Z_\odot=0.016$.} }


\section{The Subgiant Sample}\label{sec:data_selection}




% Subgiants are a useful empirical benchmark because their photospheric abundances should reflect their birth mixture. 
We use subgiant stars as our empirical benchmark.
Subgiants have the advantage that their photospheric C and N abundances should closely approximate their birth mixture.
During main sequence evolution, metals can fall below the convective envelope (i.e., ``gravitational settling''; e.g. \dbadd{CITE}).
When stars evolve off main sequence, these metals are reincorporated into the convective envelope \citep[]{gratton+00, souto19}. 
When a star becomes a red giant, \textit{first dredge up} pollutes its photosphere with core, CNO-processed material
\citep{iben67, KL14}.

We use the sample of \apogee\ DR17 subgiants from \citet{jack}. We select stars from the ASPCAP pipeline (revision 1) within the $\log g$-$T_\text{eff}$ polygon given by
 \begin{equation}
    \begin{cases} \label{eq:subgiant_selection}
        \log \text{g} \geq 3.5 \\
        \log \text{g} \leq 0.004\,T_{\rm eff} - 15.7 \\
        \log \text{g} \leq 0.0007\,T_{\rm eff} + 0.36 \\
        \log \text{g} \leq -0.0015\,T_{\rm eff} + 12.05 \\
        \log \text{g} \geq 0.0012\,T_{\rm eff} - 2.8. \\
    \end{cases}
\end{equation}
Following \citet{jack}, we also exclude stars with the flags:
        \verb|APOGEE_MIRCLUSTER_STAR|,
        \verb|APOGEE_EMISSION_STAR|,
        \verb|APOGEE_EMBEDDEDCLUSTER_STAR|,
        \verb|APOGEE2_YOUNG_CLUSTER|,
        \verb|APOGEE2_W345|,
        \verb|APOGEE2_EB|, and
        \verb|DUPLICATE|.
We additionally remove stars with no reported (or flagged unreliable), C, Mg, or Fe abundances. These cuts yield a sample containing \nsubgiants\ subgiants.


Fig.~\ref{fig:subgiants} shows out subgiant sample plotted in the \caah\ and \caafe{} planes.\footnotemark{} In this sample, [C/Mg] increases with [Mg/H], and [C/Mg] decreases with [Mg/Fe] at fixed [Mg/H]. 


\footnotetext{In this paper, we use the standard notation for chemical abundances. $[A/B] = \log_{10}\left(A/B\right) - \log_{10}\left(A_{\sun}/B_{\sun}\right)$, i.e. $[A/B]$ is the logarithm of the mass ratio between A and B, scaled such that $[A/B]=0$ for the sun (see Table~\ref{tab:fiducial_mod} for solar scale.) }

We have also investigated these thrends in other catlogues (not show) in order to understand the systematic uncertainties of our APOGEE subgiant sample.
Namely, we we consider subgiants/dwarfs from GALAH DR4, Gaia-ESO DR5.1, and LAMOST with similar selection criteria. Additionally, we consider APOGEE Giants with first dredge up mixing corrections from \citep{vincenzo+21}.
In general, different surveys follow similar trends in \caah\ and \caafe\ but with systematic offsets at the $\sim{0.1}$ dex level in [C/Mg] given [Mg/H] and slight variations in the slope.
\dbnote{Galah DR4 fixes the issues we were seeing and updates how they process C.}
These systematic differences are not a major concern for our conclusions since our models are mostly constrained by the slope of \caah\ and \caafe,  not their normalization (see Fig.~7? and discussion in section \dbadd{5.2}).

\section{Nucleosynthesis}\label{sec:nucleosynthesis}

Following \citet{james+23}, we integrate yields and our GCE models using the publicly available Versatile Integrator for Chemical Evolution (\VICE, \citealt{JW20}). We retain their prescription for O, Fe, an yields, scaled down to the normalization recommended by \citet{david_fe} based on .....
Table~\ref{tab:fiducial_mod} summarizes these choices. As required by \VICE, these yields quantify the mass of some element that is newly produced (i.e. net yields as opposed to gross yields). The SN yields are in units of the proginitors initial mass, while the AGB yields are defined relative to the star's zero-age main sequence mass.

\dbadd{paragraph about yields in general}
A yield quantifies the new production of a chemical species. For a given star, we define the {\it net fractional yield} to be the mean change in the chemical abundance from the birth to ejected mass, multiplied by the ejected fraction of a stars mass, i.e.
\begin{equation} 
   Y_{\rm C}^{\rm AGB}(Z) = 
    \delta Z_X \frac{M_{\rm ej}}{M}
\end{equation}
Note that yields may become negative if more of the initial chemical abundance of X is transformed into other elements than other elements transformed into X.
In general, we describe yields by their population-averaged forms, the amount of newly produced X per unit mass of star formation (from supernovae and stellar winds).
\begin{equation} 
   y_{X}(Z) = 
    \frac{M_X - M_{\rm X, ini}}{M_{\rm ssp}}
\end{equation}

In \VICE, massive star nucleosynthesis is approximated as occurring instantaneously after a star formation event. Under this approximation, the yield is simply a constant  of proportionality between the metal production rate and the SFR:
\begin{equation}
    \dot{M}_X^{\rm CC} = y_{X}^{\rm cc}\, \dot{M}_\star
\end{equation}
The population-averaged yield, $y_{X}^{\rm cc}$ can be expressed as
\begin{equation}
y_{X}^{\rm CC} = \frac{
\int_8^u (E(m) m_X + w_X - Z_X(m - m_{\rm rem}) )\frac{dN}{dM}\ dM
}{
\int_l^u m\,\frac{dN}{dM}\ dm
}
\end{equation}
where $m_X$ and $w_X$ are the mass of X ejected in a CCSN explosion and through stellar winds respectively, from a star of initial mass $m$. $E((m)$ is a function describing the fraction of stars of mass $m$ that successfully explode (se discussion in \S \dbadd{3.2} below). $m_{\rm rem}$ is the mass of the remnant left behind, $dN/dM$ is the IMF, for which we assume \dbadd{cite}. $l$ and $u$ are the lower and upper mass limits of star formation. $8\,\Mo$ is the minimum mass of a CCSN progenitor. In section \S 3.2 below, we compute the values of $\Ycc$ using \VICE's `vice.yields.ccsne.fractional` function and discuss theoretical expectations of C yields based on massive star evolutionary models. 


The population averaged SNe Ia yield, $y_{X}^{\rm ia}$, is given by,
\begin{equation}
    y_{X}^{\rm Ia} = m_{x}^{\rm Ia} \frac{N_{\rm Ia}}{m_\star}
\end{equation}
where $m_{X}^{\rm Ia}$ is the average mass of X produced in a single SN Ia event, and $N_{\rm Ia} / M_\star$ is the number of such events per unit mass of star formation. The latter can be expressed as an integral over the SN Ia delay-time-distribution (DTD), which describes the rate at which stellar populations produce X through SN Ia in \VICE. Following \citet{james+23}, we use a $t^{-1.1}$ single power-law with a minimum delay-time of 150 Myr based on comparisons of the SN IA rate as a function of redshift and the cosmic SFR (e.g. \citealt{maoz+12}). 

The net yield from a single AGB star is given by \begin{equation} \label{eq:net-yield}
   Y_{\rm C}^{\rm AGB}(Z) = 
    \delta Z_X \frac{M_{\rm ej}}{M}
\end{equation}
where $M_{\rm ej}$ is the mass of the ejected envelope, and $\delta Z_X$ is the average change in the abundance of X in the envelope over the lifetime of the star. Models of AGB star nucleosynthesis report values of $Y_X^{\rm AGB}$ sampled on a grid of initial masses and metallicities (see discussion in \S \dbadd{3.1}), and \VICE\ accepts these tables directly as input. The population-averaged AGB star yield is then given by 
\begin{equation} \label{eq:imf-agb-yield}
    y_{X}^{\rm agb}(Z,\tau) = 
    \frac{
    \int_{l}^{M_{\rm max}^{\rm AGB}} 
    \y_{\rm X}^{\rm proc}(M, Z) M\ 
    \frac{dN}{dM}\ dM
}
{
    \int_{l}^{u}\ M\ \frac{dN}{dM}\ dM
}
\end{equation}
where $m_{\rm min}$ is the minimum and maximum initial mass of stars that become AGBs. While $m_{\rm min}$ theoretically extends as low as $0.1 \Mo$, we interpolate yields to zero past $0.5 \Mo$ as these stars live longer than the disk and no yield grid extends this low. In Section~\ref{sec:agb},  we discuss theoretical expectations for C production in AGB stars and our adopted tables of $\Ycagb$ from the literature.
\dbadd{specify exactly how we change the yield interpolation for AGB stars}



\begin{table}
	\centering
    \caption[]{Solar abundance scale and fiducial yields (in units of stellar population birth mass). See section \ref{sec:agb} for the definition of \fruity. The solar abundance scale is \citet{magg+22} + 0.04. }
	\label{tab:fiducial_mod}

	\begin{tabular}{l l l l l}
		\hline
        Element & $Z_{X,\,\sun}$ & $y_X^{\rm cc}$ & $\y_X^{\rm agb}$ & $y_X^{\rm ia}$  \\
		\hline
        C & 0.00339 & Eq.~\ref{eq:y_cc} & $\cfactor\times$\fruity &  0 \\
        O & 0.00733 & 7.12e-3 & 0 & 0 \\
        Mg & 0.000671 & 6.52-4 & 0 & 0 \\
        Fe & 0.00137 & 4.72e-4 & 0 & 6.69e-4 \\
        N &0.00104 & 4.001e-4 & 0.0009$M\left(\frac{Z}{Z_\odot}\right)$ & 0\\
        % C & 0.00339 & Eq.~\ref{eq:zeta} & $\cfactor\times$\fruity &  0 \\
        % O & 0.00733 & 0.015 & 0 & 0 \\
        % Mg & 0.000671 & 0.00185 & 0 & 0 \\
        % Fe & 0.00137 & 0.0012 & 0 & 0.00214 \\
        % N &0.00104 & 0.00072 & 0.0009$M\left(\frac{Z}{Z_\odot}\right)$ & 0\\
		\hline
	\end{tabular}
\end{table}

\subsection{Asymptotic Giant Branch Stars}\label{sec:agb}


\begin{figure*}
    \centering
 	    \includegraphics[scale=1]{agb_yields.pdf}
        \caption[]{\dbnote{replace y label with stellar C yield, draw actual interpolations...}. The net fractional \agb\ C yield  plotted as a function of initial stellar mass $M$ and colour-coded according to metallicity. The black dashed line shows $\y=0$ for reference. Each panel represents yields from one of four \agb\ models: \fruity{}, \aton{}, \monash{}, \nugrid{} (see sections \ref{sec:agb}) }

        \label{fig:y_agb}
\end{figure*}

\begin{figure*}
    \centering
    \includegraphics{y_agb_vs_z.pdf}
    \includegraphics{y_agb_vs_t.pdf}

    \caption[]{\textbf{Left} \dbnote{ylabel: integrated carbon AGB yield} The (IMF-weighted) \agb\ C yield $\Ycagb$ as a function of metallicity for each of the \agb\ yield models. ($\Ycagb$ is the net mass of C produced by \agb\ stars per unit mass of star formation, after 10\,Gyr and assuming a \citealt{kroupa01} IMF.)
    \textbf{Right} cumulative C return as a function of age for a solar metallicity single stellar population. The dashed black line shows the delay time distribution of type Ia supernovae ($\propto t^{-1.1}$) for comparison. The minimum of \aton{} is at $y( 0.30\,{\rm Gyr})/y(t_{\rm end})=-7.81$ and is large as the net yield is close to zero.
}

    \label{fig:agb-ssp}

\end{figure*}


An Asymptotic Giant Branch (AGB) star is a low mass star during its last nuclear (He shell) burning phase which enriches the interstellar medium (ISM) through mass loss during thermal pulses.(see e.g. \citealt{PR2023}). 
In this work, we explore four different stellar AGB yield tables from the literature which provide well-sampled grids in mass and metallicity. We refer to the tables as 
\begin{description}
    \item \hypertarget{fruity}{\texttt{FRUITY}}: \citet{cristallo+11, cristallo+15}
    \item \hypertarget{aton}{\texttt{ATON}}: \citet{ventura+13,ventura+14,ventura+18, ventura+20}
    \item \hypertarget{monash}{\texttt{Monash}}: \citet{KL16, karakas+18}
    \item \hypertarget{nugrid}{\texttt{NuGrid}}: \citet{pignatari+16, ritter+18, battino+19, battino+21}
\end{description}
Table~\ref{tab:agb} notes the masses and metallicities for each grid of yields.
We use the same tables as \citet{james+23}, except that we have swapped \citet{karakas10} for \citet{pignatari+16}. \dbadd{the selection of tables is the same as CITE with the exception that ...}
For our models to better match observations, we uniformly amplify the yield tables according to
\begin{equation} \label{eq:alpha}
        \Ycagb \rightarrow \alpha_{\rm agb}\ \Ycagb.
\end{equation}
We use the \fruity\ table, with $\alpha_{\rm agb}=\cfactor$, as the fiducial \agb\ yield.

Fig.~\ref{fig:y_agb} compares the C yields for these four AGB models.
\dbadd{there is good agreement} Most models agree on general yield trends in mass and metallicity.
C production peak between masses of $\sim$ 2--4 $\Mo$. As metallicity increases, the net yield decreases and the mass of peak C production increases slightly.

In order to integrate yields across the metallicity and mass range of interest, we interpolate yield tables linearly in both $Z$ and mass. Below the last sampled mass, $Y_{\rm C}^{\rm AGB}$ is linearly interpolated to 0 at $0.5\Mo$. The choice of yields for stars of masses $\Mo \lesssim 1.5$ produces about a 20\% uncertainty in the slope of the DTD past 3\,Gyr \dbstrike{but these old populations account for a small fraction of the total C}. \strike{More complete grids in mass and metallicity would be needed to more accurately compare GCE predictions between AGB evolution models. }

The left panel of Fig.~\ref{fig:agb-ssp} shows IMF-averaged C yields for each \agb\ model as a function of metallicity.
The IMF-averaged yields differ in normalization and metallicity dependence.  
The normalizations span a factor of \about{2}. (e.g. between $6\times 10^{-4}$ and $12 \times 10^{-4}$ at $Z=0.33\,\Zo$).
Both \nugrid{} and \aton{} predict non-monotonic C yields with $Z$, but it is unclear if this prediction is significant given the uncertainties associated with any one model.
To characterize the metallicity dependence of each table, we fit the yield to a linear model in $\log Z$
\begin{equation}\label{eq:agb_z_approx}
    y_{\rm C}^{\rm AGB}(Z) = y_0 + \zagb \log(Z / \Zo).
\end{equation}
Table \ref{tab:agb} contains the fit parameters $y_0$ and $\zagb$ for each AGB table.
The values of $\zagb$ span a factor of $\sim{3}$ between models.
These variations are due to different choices of reaction rates, convection treatments, and mass-loss rates \citep{ventura+15} \dbadd{REF}, but can be qualitatively understood by comparing the strength of third dredge up (TDU) and hot bottom burning (HBB) (see discussion below and \JJ). 



The right panel of Fig.~\ref{fig:agb-ssp} shows the total production of C by \agb\ stars by a single stellar population (SSP; cluster of stars born at the same time from the same material) as a function of age at solar metallicity. 
As the mass range $2\,\Mo\lesssim M \lesssim 4\,\Mo$ is predicted to be most important for C production, half the yield is produced before \about{1}\,Gyr, similar to Fe production by \ia. 
\monash{} weight C production more heavily towards high-mass \agb\ stars, resulting in shorter delay times, whereas the \fruity\ and \aton\ models predict a slightly longer timescale of \about{1}\,Gyr. In any case, little to no C is produced more than 2\,Gyr after a star formation event. Fe production, in contrast, continues steadily for 10\,Gyr. 

In AGB stars, {third dredge up} (TDU) and {hot bottom burning} (HBB) are two processes driving C evolution in \agb\ stars with competing effects.
TDU accompanies thermal pulses, where material from the CO core is mixed with the envelope, increasing surface C abundances 
later released to the ISM \citep{KL14}.  
HBB\ is the activation of the CNO cycle at the bottom of the convective envelope \dbadd{with a net effect of ...}.
TDU increases C yields and HBB converts C into N. When both processes are active, highly efficient N production ensues (see discussion in e.g. \citealt{james+23, ventura+13}). 
%Because the $^{14}$N proton capture is the slowest component of the CNO cycle, the CNO cycle converts nearly all \C[12] into $^{14}$N \citep{solar-fusion}.

% \footnotetext{
%     The CNO cycle is a series of proton-capture reactions with CNO elements resulting in energy generation and the creation of an $\alpha$ particle. $\C[12]({\rm p}, \gamma)
%     ^{13}{\rm N}(\beta^+, \nu_{\rm e})
%     ^{13}{\rm C}({\rm p}, \gamma)\allowbreak
%     ^{14}{\rm N}({\rm p}, \gamma)\allowbreak
%     ^{15}{\rm O}(\beta^+, \nu_{\rm e})\allowbreak
%     ^{15}{\rm N}({\rm p}, \alpha)
%     \C[12]$. 
% There are other less important minor branches of the CNO cycle
%  \citep{solar-fusion}.
% }


Both HBB and TDU result in mass- and metallicity-dependent C yields. 
Stars less than \about{2}\,$\Mo$ do not experience TDU. As a result, C yields from these stars are affected only by first dredge-up \citep[Table 1]{karakas10}, resulting in small net C yields.
Above \about{2}\,$\Mo$, TDU becomes important, enriching the outer layers with C.
In \agb\ stars more massive than \about{5}\,$\Mo$, efficient HBB turns most \C[12] into $^{14}$N.
TDU is more efficient for more compact, massive cores. HBB requires a more massive AGB star to reach sufficient temperatures at the base of the convective envelope to initiate proton captures. Both processes are more efficient at low metallicity due to the lower opacities and, more compact internal structures, and hotter temperatures. These two processes drive C yields from AGB stars, but their details are ultimantly set by the uncertain choices for stellar evolution models.

% compactness causes stronger thin-shell instability
% C yields increase because TDU outweights HBB when low Z.


The major uncertanties in AGB stellar evolution are convection, nuclear reaction rates, and mass loss prescriptions.
In particular, the $^{14}{\rm N}({\rm p},\gamma)$ reaction rate uncertainty causes a factor of 2 difference in predicted C yields \citep{herwig+austin2004, HAL2006} \dbnote{look at present-day reaction rates here}.
CITES compared the \aton{} and \monash{} models, finding that differences in yields are primarily driven by differences in mass-loss and convection prescriptions. \aton{} predicts strong HBB to set in $\sim{1}\,\Mo$ lower than in \monash{}, leading to much lower C yields.
Most of the physics included in stellar modeling is empirically calibrated. Mass-loss rates, convection, and extra mixing are not sufficiently well understood to be computed from first principles.




\begin{table*}
	\centering
    \caption[]{For each \agb\ yield set, the IMF-averaged \agb\ C yield at solar metallicity $y_{\rm C, 0}^{\rm agb}$. The solar value and slope for a linear fit to the yield (least squares, excluding lowest-metallicity point for \aton). The fraction of solar C produced in the model $f_\odot^{\rm AGB}$.  We also include the masses and metallicities each grid is sampled on.
    \dbnote{MCMC gives same results as LS here, so we can simplify this discussion.}
    }

	\label{tab:agb}
    \begin{tabular}{c  ccc  c p{5cm} p{5cm}} % four columns, alignment for each
		\hline 
        \agb\ table 
                & $y_{\rm C}^{\rm agb}(\Zo)\times10^4$ %& y
                & $\bar y_{\rm C}^{\rm agb}(\Zo)\times10^4$ %& y
                & $\zagb(\Zo)\times10^4$ %& $\zeta$ 
                &  $f_\odot^{\rm agb}$
                & masses ($\Mo$) & metallicities ($Z$)\\
        \hline
        \fruity 
                & 3.229
                &  $3.8\pm0.3$
                & $-3.5\pm0.3$
                & 0.138
                & 1.3, 1.5, 2, 2.5, 3, 4, 5, 6
                & 0.0001, 0.0003, 0.001, 0.002, 0.003, 0.006, 0.008, 0.01, 0.014, 0.02
                \\
        \aton 
                & 0.2851
                & $1.4\pm1.8$
                & $-10. \pm 3$
                & 0.067
                & 1.5, 2, 2.5, 3, 3.5, 4, 4.5, 5, 6, 6.5, 7
                & 0.0003, 0.001, 0.002, 0.004, 0.008, 0.014, 0.04
                \\
        \monash 
                &  3.444
                & $2.8 \pm 0.5$
                & $-11.0\pm 1.0$
                & 0.102
                & 1, 1.25, 1.5, 1.75, 2.25, 2.5, 2.75, 3, 3.25, 3.5, 3.75, 4, 4.5, 5, 5.5, 6, 7 
                & 0.0028, 0.007, 0.014, 0.03
                \\
        \nugrid 
                & 10.95
                & $8.3\pm 1.9$
                & $-4.6\pm1.7$
                & 0.214
                & 1, 1.65, 2, 3, 4, 5, 6, 7
                &  0.0001, 0.001, 0.006, 0.01, 0.02
                \\
		\hline
	\end{tabular}
\end{table*}




\subsection{Core Collapse Supernovae}


Massive stars form $^{12}$C in their cores through the triple--$\alpha$ reaction \dbnote{any more introduction needed here?}. Later, C not transformed into other elements is released to the ISM through winds or a supernovae.
While there are many stellar models providing predictions of \cc{} yields, the results of these models are highly uncertain due to the complexity of stellar modeling. \dbnote{maybe (re)move this sentence? not sure...}

Fig.~\ref{fig:y_cc} plots calculations of IMF-averaged yields for a handful of massive star yields from the literature.
\cc{} models predict a wide range of C yields, spanning nearly a factor of ten. 
Rotation, binarity, and explodability introduce substantial uncertainties in \cc\ predictions \citep{farmer+21}. The \cite{LC18} models, which include rotation, show that the induced mixing (e.g. \citealt{frischknecht+16}) can dramatically increase the magnitude and metallicity dependence of $\Ycc$. As we will later emperically show, \cc\ C production needs to be strongly metallicity-dependent at $Z \approx \Zo$, which is consistent with \citepos{LC18} rapidly rotating models and to a lesser extent \citet{NKT13}.
As metallicity increases, stars lose more of their mass to winds. In particular, C enriched envelop material is lost through winds before synthesized into heavier elements, resulting in $Z$-dependent C production \citep{LC18}.
The left panel of Fig.~\ref{fig:y_cc} shows the \fruity{} \agb\ model. Especially at $Z\approx Z_\odot$, most \cc models dominate \agb\ C production. 


The right panel of Fig.~\ref{fig:y_cc} shows the average [C/Mg] ratio of \cc\ ejecta for the different models, defined by
\begin{equation}\label{eq:c_mg_cc}
    {\rm [C/Mg]^{CC}} = \log_{10}\left( \frac{\Ycc}{\Yoc}\right) - \log_{10} \left( \frac{Z_{{\rm C},\ \sun }}{Z_{{\rm Mg},\ \sun }} \right).
\end{equation}
The models we have considered here predict [C/Mg]$^{\rm CC}$ ratios that closely trace the metallicity-dependence of the C yield, a consequence of approximately
metallicity-independent Mg production \citep[e.g][]{andrews+17}.
However, [C/Mg]$^{\rm CC}$ is super-solar in all models except
\citet{NKT13}.
This result arises due to the so-called ``oxygen-magnesium problem,'' whereby Mg
is under-produced relative to O (see discussion in \citealt{emily+21}).
resulting in [C/Mg] ratios higher than observed.
To avoid this problem in our \gce\ models, we simply assume the O and Mg yields
from massive stars reflect the solar mixture, consistent with observations from \apogee\ \citep{weinberg+19, weinberg+22} \dbnote{update these citations?}.


To simplify the exploration of C yields from \cc\, we use the parameterazation, 
\begin{equation}\label{eq:y_cc}
    \Ycc = \begin{cases}
    y_{\rm C, 0}^{\rm cc} + \zcc\,\log(Z/\Zo) + A \log(Z/\Zo)^2 & Z \geq Z_{\rm low}
    \\
    y_{\rm C, low}^{\rm cc} & Z < Z_{\rm low}
    \end{cases}
\end{equation}

where $Z_{\rm low}$ is the transition between low-metallicity constant yields and the linear(quadratic) yields near solar. We take $A=0$ in the fiducial model, representing a linear model, and $y_{\rm C,low}^{\rm CC} = 0.XXX$ with $Z_{\rm low} = XXX$. This results in a continuous yield where 
In our Quadratic model, we take $Z_{\rm low}$ to be the vertex of the parabola, such that the yield is monotonic and $y_{\rm low}$ is the value of the vertex.
\dbnote{Fill in description here. Could also try a BiLinear model with vertex near Fe/H=0}
\begin{figure*}
    \centering
    \includegraphics{cc_yields.pdf}
    \caption[]{
    \dbnote{y label: integrated CCSNe carbon yield}
    \dbnote{might be interesting to plot every CCSNe yield used in this paper in light grey to show strength of our constraint}
        C yields from high-mass stars.
        \textbf{Left} The IMF-weighted \cc\ yield of C as a function of metallicity.
        \textbf{Right} The \cc\ [C/Mg] abundance ratio, defined in Eq.~\ref{eq:c_mg_cc}. The black line is the derived C yield from section \ref{sec:equilibrium} and Eq.~\ref{eq:y_cc}. Yields are shown for tables from 
    \citet[red triangles]{WW95}, \citet[orange square and diamond]{sukhbold+16}, 
    \citet[green stars]{NKT13}, and \citet[blue circles]{LC18}. \citet{sukhbold+16} report yields for different black hole landscapes, while \citet{LC18} provide yields at different rotational velocities.
    In the left panel, the pink line denotes $\Ycagb$ from \fruity{} for comparison. All models include wind yields. 
}
    \label{fig:y_cc}
\end{figure*}



\subsection{Adpoted Fiducial Yields}
We propose the following functional form for total C yields near solar metallicities:
\begin{subequations}\label{eq:yc_inferred}
    \begin{align}
        \frac{\Yct(Z)}{y_{\rm Mg}} &\approx 4.20 + 1.64 \log (Z/\Zo)\\
        \frac{\Yct(Z)}{y_{\rm Mg}} &\approx 4.22 + 2.178 \log(Z/\Zo) + 5.53 \log(Z/\Zo)^2
    \end{align}
\end{subequations}
At solar metallicity, these yield ratios results in an equilibrium abundance $[{\rm C/Mg}] = -0.08$ which is consistent with our subgiant sample and is within \about{20\%} of the solar C/Mg mixture. 
For the quadratic form, we hold the yields constant past the vertex (at {\color{red} [M/H] = ?}) to avoid a premature upturn in \caah.
We discuss the choice of these functional forms more in appendix~\dbadd{X}.



\dbnote{check zetas}
With the total C yield of Eq.~\ref{eq:yc_inferred} and given an \agb\ C yield, we can derive an \cc\ C yield that would predict observationally consistant [C/Mg] ratios.
(by subtracting the assumed AGB contribution from the total yield fit above.)
\begin{subequations}
    \begin{align}
        \Ycc(\Zo) &= \Yct(\Zo) - \Ycagb(\Zo)\\
        \zeta &= \zcc + \zagb
    \end{align}
\end{subequations}
% \dbnote{I think I can get rid of the below equation.}
% \note{I'd suggest keeping it -- since you use $\zeta$ to distinguish between models, and it shows up in figure legends, the visibility of its exact definition will improve readability.}
where $\zeta$ is the (solar) metallicity dependence of the C yield at $Z=\Zo$
\begin{equation}
    \zeta = \frac{d y_{\rm C}}{d \log Z} \Big \vert_{(Z=\Zo)},
\end{equation}
and $\zcc$ and $\zagb$ correspond to the specific enrichment channels. 

Both observational and theoretical uncertainties limit the accuracy of our relative yield predictions. Additionally, the derived yields will be systematically biased if the galaxy is out of equilibrium, e.g. due to a recent starburst \citep{mor+19,isern19}. 


\section{The Multi-zone Model}\label{sec:vice}

Our models extends the \citet[hereafter \JJ]{james+21} Milky Way model, run with the publicly available Versatile Integrator for Chemical Evolution (\VICE).%
    \footnote{\VICE~is available at \url{https://github.com/giganano/VICE}}
This model is described extensively in \JJ~and concisely summarized  in \citet{james+23}. Here, we provide a brief overview of the relevant model components.

Classical, \textit{one-zone} models of chemical evolution assume instantaneous mixing of metals in the star-forming ISM\ \citep[e.g.][]{matteucci21}. This simple framework is a poor approximation of the Milky Way.  The Galaxy evolves \textit{inside-out} -- where star formation is higher towards the center and in the early universe \citep{WF91, kauffmann96, bird+13}. Stars can also migrate several kpc over their life/times, mixing different chemical environments across the galaxy \citep{bird+12,sellwood+binney02}. Multi-zone models account for stellar migration and changing environments by stitching together multiple one-zone models and mixing stellar populations. \dbnote{does this belong in intro or here?. Thinking here now because feels more like methodology justification, but not sure still.}

For our O, Mg, and Fe yields, we adopt the yields in \citet{david_fe}. We use a scaled variation of the AGB N yield from \citet{james+23} with a metallicity-independent CCSNe N yield. 
Following \citet{james+21, james+23}, we take the \ia{} delay time distribution to be a $t^{-1.1}$ power-law with a minimum delay time of 150\,Myr, as suggested by the observations of \citet{maoz+12}. 
\dbnote{discuss SNe Ia (maximum 4\% of C but likely much less) and SAGB (also minimal). Also: can novae produce any C?}


The Galaxy is divided into 200 rings, each representing a single, 100\,pc zone. Each ring (or zone) has a separate stellar population and gas supply. We initially assume an inside-out SFH from \JJ, where the star formation surface density $\dot{\Sigma}_\star$ is given by 
\begin{equation}\label{eq:inside_out}
    \dot{\Sigma}_\star \propto \left(1-e^{-t/\tau_{\rm rise}}\right) e^{-t/\tau_{\rm sfh}}.
\end{equation}
$\tau_\text{rise}=2$\,Gyr loosely describes when the star formation rate reaches a maximum, and $\tau_{\rm sfh}$ describes the decay timescale of star formation as a function of radius $R$. \JJ\ derive $\tau_{\rm sfh}(R)$ through analysis of four integral field spectroscopy surveys in \cite{sanches20}. At each $R$, the SFH is normalized to match the stellar surface density gradient \citep{BHG16} assuming a total stellar mass of $5.17\times10^{10}\,\Mo$ \citep{LM15}. Star formation ends beyond a radius $R=15.5\,$kpc, but stellar populations are allowed to migrate as far as $R=20\,$kpc.  
The gas inflow is calculated to maintain the SFH for each radius and time, using an extension of the Kennicutt-Schmidt law \citep{kennicutt98} motivated by the combined observations of \citet{bigiel+10} and \citet{leroy+13}, 
\begin{equation}
\dot{\Sigma}_{\star} \propto 
\begin{cases}
    {\Sigma}_{\rm gas} & 2\times 10^7 \Mo\,{\rm kpc}^{-2} \leq \Sigma_{\rm gas} \\ 
    {(\Sigma}_{\rm gas})^{3.6} & 5\times 10^6 \Mo\,{\rm kpc}^{-2}\leq \Sigma_{\rm gas} < 2\times10^7 \Mo\,{\rm kpc}^{-2}\\ 
    {(\Sigma}_{\rm gas})^{1.7} & \Sigma_{\rm gas} < 5\times10^6 \Mo\,{\rm kpc}^{-2}.
\end{cases}
\end{equation} 
The scaling of this relationship varies with time due to the redshift dependence of $\tau_\star$ in molecular gas observed by \citet{tacconi18}. We assume a \citet{kroupa01} IMF.


To account for radial migration, we use a normally-distributed, $\sqrt{\rm time}$ migration scheme. The final position of each star particle is sampled from
\begin{subequations}
\begin{align}
        R_{\rm end} &\sim N(R_{\rm birth}, \sigma_R ) \\
        \sigma_{R} &= 1.27\,{\rm kpc} \sqrt{\frac{t_{\rm end} - t_{\rm birth}}{1 \rm Gyr}}
\end{align}
\end{subequations}
If a star would cross a boundary, then the star is reflected back.
The star moves from its birth radius to its final radius via
\begin{equation}
        R(t) = \Delta R \sqrt{\frac{t - t_{\rm birth}}{t_{\rm end} - t_{\rm birth}}}
\end{equation}
The $\Delta R \propto \sqrt{\rm time}$ dependence arises when migration proceeds as a consequence of the diffusion of angular momentum \citep{frankel18, frankel20}.
At each time step, each star particle travels a radial distance sampled from 
where $N(\mu, \sigma)$ represents a draw from normal distribution and $dt=20$\,Myr is the simulation timestep.%
\footnote{Not shown here, we also explored variations of temporal, spatial resolution, migration strength, and the mass lifetimre relation. We found these do not affect median trends significantly.}
We do not account for radial gas flows.
Not shown here, we also explore migration based on random walks and the \texttt{h277} hydrodynamical
simulation results\footnote{(with simulation parameters as in \citealt{bird+21}; see also \citealt{christensen12, zolotov12, loebman12, BZ14} \dbnote{is this okay as footnote?} }, which leaves our qualitative conclusions unchanged. 
The full impact of the details of a galaxy's dynamical history on its chemical evolution is still unknown.

As the strength of outflows controls the resulting $\alpha$-element abundances, we extend \JJ and create a metallicity gradient by defining
\begin{equation}\label{eq:mass_loading}
\eta(R) = \frac{y_{\alpha}^{\rm cc}}{Z_{\alpha}(R)} -1 + r + \tau_{\star} / \tau_{\rm sfh} 
\end{equation}
where 
\begin{equation}
    \log Z_{\alpha}(R) = \log Z_{\alpha,\ \odot} + 
    0.29 + 
    \begin{cases}
        -0.015(R-5) & R < 5 \\
        -0.09(R-5) & R \geq 5
    \end{cases}
\end{equation}
is adapted from \citet{hayden+14}.  We approximate $r=0.4$, and $\tau_\star/\tau_{\rm sfh} \approx 0.0$ \dbnote{double check this...}.
This choice of $\eta(R)$ results in a [$\alpha$/H] gradient consistent with Milky Way observations \citep[e.g.][]{hayden+14, weinberg+19, frinchaboy+13}.
If we change our assumed $y_{\rm Mg}$, the values of $\eta$ will change similarly to maintain the correct chemical trends.


Finally, we tune the carbon yields of CCSNe in slope and normalization to approximately match the median trends in most cases.
A tabulated description of our model parameters is available in appendix~\dbadd{model parameters}.

\section{Results}\label{sec:results}

\subsection{Evolution of Carbon Abundances}


\begin{figure*}
\centering
\includegraphics{figures/all_the_tracks.pdf}
\caption[]{
    Time evolution of gas-phase C abundances in our fiducial model for [C/Mg] versus [Mg/H] (left) and [Mg/Fe] (right).
    Each line represents a zone at a different galactic radius and is colour-coded by time. While \caah\ settles into the final trend $\sim{8}$\,Gyr ago, \caafe\ continues to evolve for much longer.
}
\label{fig:c_evo}
\end{figure*}



Fig.~\ref{fig:c_evo} shows evolutionary tracks for every zone of the fiducial model for [C/Mg] against [Mg/H] and [Mg/Fe].
\caah\ trend quickly reaches our equilibrium state in \about{5}\,Gyr. \caafe\ tracks continue to evolve due to the long tail in the DTD of \ia. 


The evolution of abundances in our fiducial model:
\begin{enumerate}
    \item[(1)] Star formation begins. Initially, \cc\ dominate enrichment. [C/Mg] evolves with yields set by $\Ycc/\Yoc$, resulting in increasing values with [Mg/H].  [Mg/H] quickly approaches its equilibrium value, but Fe remains low in comparison (in agreement with \citealt{WAF17}).
    \item[(2)]  A few hundred million years later,  delayed sources  enrich the ISM. \agb\ stars release C, raising [C/Mg], and \ia\ expel Fe, lowering Mg/Fe. 
    \item[(3)] Several billion years later, [C/Mg] plateaus as C also approaches equilibrium. [Mg/H] reaches equilibrium. [Fe/H] continues to incrase due to ongoing \ia\ from old stellar populations.
    \item[(4)] In the final few billion years, [C/Mg] decreases slightly due to declining \agb\ C yields. \caah\ trends shift only slightly, and [Mg/Fe] slowly and steadily decreases.

\end{enumerate}


This evolution is driven most dominantly by our chosen elemental yields, their
dependence on metallicity, and their DTDs.
As we will demonstrate in section 5.2 below, the slope of the [C/Mg]-[Mg/H] trend is
set by the relative contributions of CCSNe and AGB stars to the overall C abundance.
The positive metallicity dependence of massive star C yields outweighs the negative
dependence of AGB star yields, causing [C/Mg] to increase with [Mg/H].


To create representative stellar samples to compare with APOGEE, 
We draw \nsubgiants\ stars from the simulated stellar populations such that the selection probability is proportional to a stellar population's mass and that the subsample follows the same distribution in Galactocentric radius as the subgiants. 
Fig.~\ref{fig:equilibrium_validity} displays our sample of stars and compares them to analytic representations of the mean trend. 
The median sequence of the low-$\alpha$ stars closely follows the {\it equilibrium track}, defined by the ratio between total C and Mg yields at each metallicity. There is a slight ($\sim{0.02}$ dex) divergence between the median track and equilibrium at higher metallicities, caused by our particular choice of C yields and the finite time it takes to reach equilibrium especially at higher metallicities with an increasing yield. However, if the low-$\alpha$ sequence is approximantly matched by our {it total} C yield, then we expect our model to agree with stellar observations.
On the other hand, in \caafe\ equilibrium is a vertical line. As we assume both Mg and Fe yields are metallicity independent, there is only one value of [Mg/Fe] in the equilibrium sequence, corresponding to the low-$\alpha$ population. The median trend here instead reveals the delay time distribution of C, revealed by the green single-zone model included. 

Galactocentric radii are taken from astroNN {\color{red} citation}. \dbadd{Using Gaia parallaxes does not affect our conclusions.}



% note about \caafe plane
% As low-$\alpha$ stars were formed in regions closer to chemical equilibrium, we only plot the medians of the low-$\alpha$ sequences in \caah. 
% Instead, \caafe\ trends are set by the proportion of \agb\ C contribution. The \caafe{}-diagram, when restricted to a narrow range of metallicities, becomes an empirical delay-time-distribution of C. 
% High-$\alpha$ stars were formed in regions further from chemical equilibrium than stars with low [Mg/Fe]. Immediately after a star formation event, \cc\ elements dominate and only after sufficient time do \ia\ elements like Fe reach their higher-equilibrium abundances. 
% Likewise, high-[Mg/Fe] stars will have a greater portion of \cc\ C, as low-[Mg/Fe] stars will include more \agb\ C. 
% When binned in metallicity, median [C/Mg] changes by about 0.2 dex across the range of [Mg/Fe] at solar metallicities. As high-$\alpha$ stars have little to no delayed \ia\ Fe, these stars would also have little to no delayed \agb\ C. This means that AGB C stars make up about at most a fraction $f_{\rm agb} \approx 1 - 10^{-0.2} \approx 0.4$ of C production.
% 
% \begin{figure}
%     \centering
%     \includegraphics{apogee_caafe_binned.pdf}
%     \caption{Caption}
%     \label{fig:caafe_binned}
% \end{figure}

\subsection{Equilibrium Framework}\label{sec:equilibrium}

Based on the \citepos{WAF17} analytic models of chemical equilibrium for constant star formation, the equilibrium abundance ratio between two elements depends only on the relative yields. For C/Mg, we have \dbstrike{ ,
{james+23} demonstrate that the relationship between N and O can be
explained by the metallicity dependence of their relative yields (see discussion
in their section 4.6).
We find similar results extending these arguments to C and Mg.
The equilibrium C/Mg abundance ratio is 
}
\begin{equation}\label{eq:equilibrium_yields}
    {\rm [C/Mg]_{eq}} \equiv \log \left(\frac{\Ycc + \Ycagb }{y_{\rm Mg}}\right) - \log \left(\frac{Z_{\rm C,\,\odot}}{Z_{\rm Mg,\,\odot}}\right)
\end{equation}
\dbnote{
where  the correction for a non-constant SFH is 
\begin{equation}
    \noneqagb \equiv \frac{\dot{M}_{\rm C}^{\rm AGB}}{\Ycagb\,\dot{M}_{\star}}
\end{equation}
}



\subsection{Yield Variations}\label{sec:results_highmass}
\label{sec:agb_results}


\begin{figure*}
    \centering
    \includegraphics{figures/subgiants_equilibrium_reproduced.pdf}
    \caption{
    Similar to Fig.~\ref{fig:subgiants} except for the stars of our fiducial model. 
    The solid, black lines show the equilibrium abundance track. The black scatter points with errorbars are the median bins for the low-$\alpha$ sequence (left) and the \caafe\ trend (right). The green solid line on the right shows the singlezone evolution of \caafe\ with properties matching our solar-annulus. The median trend of low-$\alpha$ stars is an excellent approximation for the equilibrium abundance of C. 
    \dbnote{The more I look at this figure, maybe we just add green lines to Fig. 5 and skip this one}
    \dbnote{we add synthetic scatter found by fitting the metallicity-dependent reported errors with a linear fit in APOGEE and adding gaussian noise.}
    \dbnote{is there too much going on here?}
    \dbnote{would it make sense to maybe place this fig close to Fig.~1 since they are very similar so it is easier to compare? Not sure how to make continuity work then..}
    \label{fig:equilibrium_validity}
    }
\end{figure*}




\begin{figure*}
\includegraphics{sims_zeta_f.pdf}

\caption[]{
\dbnote{out of date}
    \dbnote{does it make sense to show SEM error bars instead of percentiles since we are focused on the median trend, not the spread?.}
    Stellar abundance trend predictions of our models . Coloured lines represent the median [C/Mg] in bins of [Mg/H] (left) or [Mg/Fe] (right) for each \agb\ table. Black points and grey dashes represent the median and 16th-84th percentiles of [C/Mg] in each bin in the \citet{jack}~sample. 
    The left panel only shows low-$\alpha$ stars. In the right panels, we show the trends only for stars where $-0.15\leq {\rm [Mg/H]}\leq -0.05$.
    Stars are binned into 20 (left) or 12 (right) equal number bins. 
    \textbf{Top}: Models with different metallicity dependences for  \cc\ C yields. \textbf{Bottom}: Different \agb\ fractinos of C.
}
\label{fig:zeta_f}
\end{figure*}


\begin{figure*}
\includegraphics{sims_agb.pdf}

\caption[]{
    Similar to Fig.~\ref{fig:zeta_f} except showing the mean tracks for different (unscaled) AGB yield tables with best-fit CCSNe parameters. Variations in the \caafe\ plane are due to the different total AGB C yield and variations in the mass-dependence of each yield table.
}
\label{fig:agb_predictions}
\end{figure*}


\begin{figure*}
\includegraphics{figures/zeta_f_mass_sfh.pdf}

\caption[]{
    Similar to Fig.~\ref{fig:zeta_f}, except for \caafe\ for a variety of models.
    {\bf Left} models where both the \agb\ fraction and the CCSNe slope are varied simultaneously.
    {\bf Middle} models where the \agb\ yields are shifted by some factor in mass.
    {\bf Right} alternate star formations and a model where yields and outflows are $\sim$ doubled.

    \dbnote{the left panel is the +- 3 sigma range of models from the MCMC analysis. I suspect the errors are very underestimated and I am not sure how to approach it, but the plot might make more sense if the model has the right curvature (like mass factor 0.7). There are 100+ observations per bin so maybe we need to account for apogee systematics as well??}
    
    \dbnote{I looked into the halved yields model. Since in this paper, we assume $y_o/Z_o \sim 1$, it is essentially impossible to reach solar Z when halfing the yields. I am swapping for all yields are doubled and I think if a referee complains about our choice of $\eta\sim0.4 \ne 0$ at the sun, we can deal with that later}

    \dbnote{the mass shifts change the net C production, so I may change the $\alpha$ values for these models so they are easier to compare here}
}
\label{fig:sims_degens}
\end{figure*}




The top of Fig.~\ref{fig:zeta_f} shows models with varying strengths of the $\Ycc$ metallicity dependence, $\zcc$. With higher $\zcc$, the model predicts a steeper \caah\ trend, owing to the direct relationship between the C/Mg equilibrium abundances and yield ratios. 
However, \caafe~is minimally affected by changes to $\zcc$ since \cc\ occurs on much shorter timescales than \ia\ and \agb\ enrichment.%
\footnote{The only effects on \caafe, when considering the narrow metallicity slice, are because of either the systematic change in equilibrium abundances, the imperfect evolution of the galaxy, or that the ISM abundances are set by stars which were born at poorer metallicities. }
%  Hence, \caah\ tells us about the total C yield with metallicity, which \caafe\ is independent of. If we know the \agb\ C yields, then with observed \caafe\ abundance trends, we can infer the \cc\ C yields with metallicity.



The bottom of Fig.~\ref{fig:zeta_f} shows three models with different C \agb\  fractions, defined as 
\begin{equation}\label{eq:f_agb}
    \fagb \equiv \frac{\Ycagb(\Zo)}{\Yct(\Zo)}.
\end{equation}
Modifications of $\fagb$ affect both \caafe\ and \caah. At fixed metallicity, the \caafe\ trend is principally sensitive to how much C and Fe comes from delayed sources. Increasing the \agb\ fraction therefore steepens the \caafe\ trend as expected. Though each yield model predicts the trend to flatten at ${\rm [Mg/Fe]} \lesssim 0.1 $, the median is still within the width of the observed distribution and the overall agreement is good. Additionally, increasing \agb\ contribution to C leads to a declining net slope for the total C yield owing to the negative slope of \agb\ C production with metallicity.

Because both the increase in \cc\ slope $\zcc$ and AGB fraction $\fagb$ have similar impacts to \caah, we can change both parameters simultaneously to retain agreement in \caah. We show example models in the left panel of Fig.~\ref{fig:sims_degens} in \caafe, (as models cannot be readily distinguished in \caah). The principle effect of increasing $\fagb$ and adjusting $\zcc$ correspondingly is to increase the amplitude of the trend in \caafe. Models with little to no AGB C production will produce stars with [C/Mg] independent of [Mg/Fe] in a metallicity slice. As the AGB fraction increases to an extreme $f^{\rm AGB}=0.5$, the trend becomes much steeper than the fiducial but remains a mere rescaling of the same shape. 


A second factor affecting \caafe\ is the assumed delay time distribution of \agb\ carbon production.
The middle panel of Fig.~\ref{fig:sims_degens} shows models where the AGB yields have been shifted by some factor in mass. When the AGB C production is weighted towards more massive AGB stars, then the delay time distribution becomes more like CCSN. As such, the predicted \caafe\ trend resembles the model which only contains CCSN production. The \caafe\ trend therefore is much less sensitive to the C production by stars with masses $M \gtrsim 5 \Mo$. On the other hand, shifting the C production towards low mass stars changes both the shape and amplitude of the DTD. This is evident in a transition point, where the \caafe\ trend changes slope. For the most extreme case, where the masses are shifted by a factor of 0.5; the AGB production is more delayed than SNeIa, causing a steepining of \caafe\ towards low [Mg/Fe]. Intermediate values are more linear in \caafe, matching the observed trend. We note that the fiducial model instead transitions from steep to shallow as [Mg/Fe] decreases. This is evidence that AGB C likely needs to be weighted more towards low mass stars than is predicted by current AGB stellar models.

Considering the effects of changes in the amplitude and mass of AGB C production, we can now predict and understand the differences between different yield tables.
Fig.~\ref{fig:agb_predictions} show the four C yield models (\fruity, \aton, \monash, \nugrid, with no scaling). For the most part, different \agb\ yield tables result in qualitatively similar predictions. While small adjustments in the \cc\ yields results in identical tracks through \caah, each model predicts slightly different tracks in \caafe. Differences are due to the total C production from the AGB model, and the delay time distribution of the model. For example, V13 predicts that AGB stars produce almost net zero C at solar metallicity, and \monash\ predicts the shortest delay time distribution of all models. These models thus are almost consistent with pure \cc\ enrichment of C. Additionally, \nugrid\ yields are about double \fruity, resulting in a steeper slope in \caafe.
%\aton, however, does not reproduce solar trends as the model predicts strong C destruction at slightly super-solar metallicities, resulting in inversions in both \caah\ and \caafe. 
As all \agb\ models predict C yields to decline with metallicity, all models predict an increasingly downward slope in \caafe\ with [Mg/Fe]. A recent burst in star formation may hide this downturn (see section~\ref{sec:sfh}), but C destruction at the level predicted by \aton\ is not supported by our sample. 
Because the effects of alternate \agb\ yield sets are small, we focus on \fruity{} model for the remained of this paper. Alternate yield sets minimally affect our conclusions. 


\dbadd{the inter-study differences for AGB yields can be explained by the differing mass dependence, and the metallicity dependence.}


\subsection{Variations in Star Formation History} \label{sec:sfh}
In this section, we consider an alternate SFH, namely one which exhibits a late-time elevation in the star formation rate. Motivated by the findings of \citet[see discussion in \JJ]{mor+19,isern19}, this \textit{lateburst} model
adds a Gaussian burst to the fiducial inside-out model, 
\begin{equation}\label{eq:lateburst}
    \dot{\Sigma}_\text{lateburst} \propto \dot{\Sigma}_\text{inside-out} \left(1 + A\,e^{-(t-\tau_{\rm burst})^2/2\sigma^2_{\rm burst}} \right)
\end{equation}
where $A=1.5$ represents the amplitude of the birth, $\tau_\text{burst}=10.8$\,Gyr is the time where the burst is strongest, and $\sigma_\text{burst}=1$\,Gyr is the width of the burst.

We additionally consider a two-infall like model as presented in \dbadd{dubay + 24}, which is defined as.

\begin{equation}\label{twoinfall}
\dot{\Sigma}_{\rm in} \propto exp(-t/\tau_1) + A_{2,1} \theta(t - t_{1}) \exp(-t/\tau_2)
\end{equation}

where $\tau_1=0.4\,$Gyr, $\tau_2$/G is 1/2 of the value from Sanchez, $A_{2,1}=0.5$, $t_1=4.1$, and the star formation also abruptly transitions from $\nu=2$/Gyr to $\nu=1$/Gyr at $t=t_1$.

The right panel of Fig.~\ref{fig:sims_degens} compares the fiducial, lateburst, and two-infall models. The lateburst slightly shifts \caah\ down and only slightly perturbs \caafe. These variations are small compared to the width of the observed distribution. 
Building on \citet{james+23}, we therefore conclude that the \caah\ and \caafe\ trends reflect nucleosynthetic yields more than the Galactic SFH. 

\dbnote{there has to be a good reason why \caafe\ is not affected or maybe only more extreme pertubations make a difference?}



\subsection{Degeneracies} \label{sec:outflows}

While some Milky Way \gce{} models (including ours) incorporate significant mass-loading, others
neglect mass-loading and instead use lower yields \citep[e.g.][]{MCM13, MCM14, spitoni19, spitoni20, spitoni21}.
Both classes of models are able to reproduce many details of the disk abundance structure due to the strong degeneracy between the normalizations of elemental yields and mass-loading (see discussion in, e.g. \citealt{sandford+24} and Appendix B of \citealt{james+23}). 
Our parameterization of $\eta$ illustrates this (see Equation~\ref{eq:mass_loading}) -- choosing a lower value of $y_{\alpha}$ will similarly result in lower values of $\eta$ while maintaining the same metallicity gradient. 
We consider a model where we double all yields, in the right panel of Fig.~\ref{fig:sims_degens}. A uniform change in yields and outflows leaves our median trends relatively unchanged. 

\dbnote{paragraph about how fagb and zeta are interdependent, but not fully degenerate in \caafe.}

Additionally, Fe yields and the \ia\  delay-time-distribution have their own uncertainties. Increasing both $y_{\rm Fe}^{\rm Ia}$ and $\Ycagb$ correspondingly leaves \caafe\ mostly unchanged. Not shown here, if both processes increase by a similar proportion of the total yield, then our predictions in \caafe\ are unchanged. Therefore, we cannot know the DTD of C much better than that of Fe in this framework.


\dbnote{Other things to discuss possibly ? :
 Initial mass function (and possible metallicity dependence)? Metallicity-Dependent Oxygen/Magnesium and Iron yields? Gas phase migration / ISM cooling?  MLR
 }


\subsection{Nitrogen}


\begin{figure}
\centering
\includegraphics{figures/nitrogen.pdf}

\caption[]{Similar to Fig.~\ref{fig:zeta_f} except for [C/N] against [Mg/H] for the fiducial model. Combining our results with the suggested yield from \JJ\ (rescaled to our adopted yields) explains both the thin-disk evolution of C and N. 
}
\label{fig:nitrogen}
\end{figure}

N production, also affected by the CNO cycle, is closely tied to C. As a test of our model, we use the \agb\ N yield suggested by \citet{james+23}. Fig.~\ref{fig:nitrogen} shows [C/N] versus [Mg/H] and for our fiducial model. We are able to closely match the [C/N]-[Mg/H] trend. 




\section{Gas-Phase Abundances}\label{sec:gas}

\begin{figure*}
\centering
\includegraphics[]{figures/summary.pdf}
\caption[]{
Gas-phase C abundances. 
We plot the fiducial model's present-day ISM abundances as a thick blue line (\protect\captionline[line width=1.5pt]{blue}).
Black lines are single-zone models: a fiducial yields and MW-like model (solid black), and three corrected yield models representing the MW (dotted), GSE (dashed) and Wukong (dash-dots).
Points represent measurements in 
    MW HII regions \citep[orange circles][]{mendez-delgado+22}
    extragalactic RL HII regions \citep[green circles;][]{peimbert+05, skillman+20, toribio-san-cipriano+16, toribio-san-cipriano+17, esteban+14, esteban+09}
    extragalactic CEL HII regions \citep[green diamonds;][]{garnett+95, senchyna+17, izotov+thuan99, garnett+99, berg+16, berg+19, pena-guerrero+17}
    damped Lyman-alpha (DLA) systems \citep[rust triangles;][]{omera+01, cooke+14, welsh+22, ellison+10, welsh+20, cooke+18, riemer-sorensen+17, DZ+03, cooke+17, cooke+11, dutta+14, morrison+16, srianand+10, pettini+08, kislitsyn+24, cooke+15}, 
    Milky Way stars \citep[yellow stars;][]{amarsi+19},
    and high redshift galaxies \citep[pink squares, CEL lines;][]{steidel+16, stark+14, matthee+21, mainali+20, jones+23, amorin+17, iani+23, james+14, erb+10, bayliss+14, berg+18, christensen+12, arellano-cordova+2022}.


}

\label{fig:gas_phase}
\end{figure*}
% \begin{table}
% 	\centering
%     \caption[]{Single zone model parameters. \dbnote{change dwarf parameters to match Wukong from James' dwarf paper.  $\eta=47.99\pm 5$, $t_{\rm end} = 3.36\pm 0.5$, $\tau_\star=44.97\pm 7$, $\tau_{\rm sfh} = 3.08\pm 1$}}
% 	\label{tab:singlezone_params}
% 
% 	\begin{tabular}{l l l l l}
% 		\hline
%         Model & $\eta$ & $t_{\rm end}$ & $\tau_\star$ & $\tau_{\rm sfh}$ \\
% 		\hline
%         1 & 0.5 & 13.2 & 2.5 & 14\\
%         2 & 1 & 10 & 2 & 10\\
%         3 & 9 & 3 & 5 & 3\\
% 	\end{tabular}
% \end{table}

As a final test of our model, we compare the predictions against C abundances
in the gas-phase.
These measurements are challenging since C lacks strong collisional excitation
lines, and its recombination lines fall in the ultraviolet with no nearby
H lines to reference~\citep[e.g.,][]{skillman+20}.
These two classes of spectral lines are also systematically discrepant at the
factor of~$\sim$2 level~\citep{GR07}.
As a consequence, the uncertainties associated with any one measurement are
substantial (see the representative error bars in Fig. 13), but we are able to
illuminate the underlying mean trends by combining observations from multiple
sources.
\dbnote{I believe that dust might warrent a discussion here since this can capture gas-phase metallicity, substantially affecting observations. But we also just don't know how to measure gas-phase abundances well at all. Reassuring as it means that our focus on stars is very justified.
There is an additional complication that CEL and RL abundances (especially for O) need to be put on the same scale with +0.2 dex corrections, and there is not a clear understanding which one is better.
\citet{MM19}
}

\dbnote{does it make sense to include more high-redshift galaxies? 
Many of these seem to be through stacked spectra or other challenging measurments, e.g. redshifting the CEL lines in Berg++ dwarfs at $z\sim 2$
}

DLAs are .......
Most similar tothose in local group dwarfs (Cooke+2015, berg+2015, cia+2016 from MM2019 review).


Fig.~\ref{fig:gas_phase} shows our compilation alongside the gas-phase abundances in our model
at snapshots of~$t = 2$ Gyr and the present day.
While we used Mg as our representative alpha element in our APOGEE sample, O
is more readily observed in the gas-phase, so we shift focus to C/O in this
section.
The model at the present day is consistent with the mean trend seen in HII
regions in the MW and similar star forming galaxies at redshift~$z \sim 0$.
Together with our results in~\S~5, this agreement suggests that our yield model
is an accurate description of C nucleosynthesis at metallicities typical of
MW-like galaxies.

Nonetheless, our model does not extend as low in metallicity as dwarf galaxies
or (especially) damped Lyman-alpha systems.
To test our model at these abundances, we show 4 different one-zone GCE models.
\begin{enumerate}
    \item Fiducial MW: $\eta=0.5$, $t_{\rm end} = 13.2$, $\tau_\star=2$, $\tau_{\rm sfh} = 14$
    \item Corrected MW-like. 
    \item Corrected Gaia-Sausage Enceledus like. $\eta=9.56$, $t_{\rm end}=10.73$, $\tau_\star = 26.60$, $\tau_{\rm sfh} = 2.18$, infall mode
    \item Corrected Wukong-like: $\eta=47.99\pm 5$, $t_{\rm end} = 3.36\pm 0.5$, $\tau_\star=44.97\pm 7$, $\tau_{\rm sfh} = 3.08\pm 1$, infall modej
\end{enumerate}
with the exception of the fiducial like model, we use the bi-log-linear CCSNe yield model which plateaus at a [C/Mg] at low metallicity consistant with observations of metal-poor stars and dwarf galaxies.
\dbnote{check timescales to metallicity = -1 (could sagb affect the trends here?). Also try different yield sets for these just to see if any agree naturally... }

Fig.~\ref{fig:gas_phase} also plots a single-zone model and time-slices of the fiducial multi-zone models gas-phase at present day and $t=2$\,Gyr. 
Our single-zone model is designed to have parameters broadly consistent with the Gaia-Encelidus sausage.
We evolve the singlezone model for 2\,Gyr, using mass loading $\eta=20$, star formation efficiency $\tau_{\star}=30\,{\rm Gyr}$, and a star formation history $\propto e^{-t/3{\rm\,Gyr}}$.
The single-zone model uses a 1.5 mass factor times f=0.25 C11 yields. Note that the mass shift results in significant C destruction by high-mass AGB stars at low metallicities, enabling the model to reproduce the low [C/O] abundances at [O/H] = -1.


Each of these one-zone models predicts a steeper C/O-O/H trend than our
fiducial multi-ring model from previous sections.
This difference arises because the trend in the multi-ring models arises as a
superposition of end-points (see Fig.~\ref{fig:c_evo}), qualitatively similar to, e.g.,
\citepos{schonrich-binney09} argument regarding
the low-alpha sequence.
In the one-zone models, the trend instead arises as an evolutionary sequence.
To demonstrate this point further empirically, Fig.~\ref{fig:gas_phase} includes measurements
in halo and thick disk stars.
This sample indeed shows a C/O-O/H trend that is clearly steeper than our
fiducial multi-ring model, which is an accurate representation of the APOGEE
low-alpha sequence (see Fig. 8).

At the lowest metallicities ($\log_{10}(Z / Z_\odot) \lesssim -2$), our suite
of one-zone models is consistent with the C/O ratios seen in damped Lyman-alpha
systems.
For all parameter choices, these ratios simply reflect the relative C and O
yields of low-metallicity massive stars (see Fig. 5), with the increase in C/O
at higher metallicities arising due to the onset of AGB star enrichment (see
Fig. 7).
As a consequence, our yield model as parameterized predicts that C/O ratios
should never be significantly below~$\sim$half solar.
This prediction is challenged by the measurements in both dwarf galaxies and
halo/thick disk stars in Fig.~\ref{fig:gas_phase}, which suggest ratios~$\sim$$0.2 - 0.3$ dex
below solar between~$\log$(O/H) = X and Y.

This discrepancy could arise at least in part due to failures of our yield
prescription and, by extension, could point to shortcomings in stellar
nucleosynthesis models.
However, we cannot rule out the possibility of systematic uncertainties in
abundance determinations playing a role.
At these metallicities, deviations from local thermal equilibrium (LTE) can
bias measurements at the~$\sim$0.2 dex level~\citep[e.g.,][]{amarsi+19}.
Considerations of non-LTE effects are not included in our APOGEE sample, but
the corrections are smaller near solar metallicity.

\dbnote{Discussion on a complex landscape of C production: elevation at lowest metallicities but with stochastic mixing, Pop III and many other problems; declining at [O/H] = -1, increasing due to Z-dependent CCSNe and delayed AGB enrichment up to 0, and flat in disks due to chemical equilibrium, but maybe also more flat CCSNe too.}


\dbnote{depending on what everyone thinks, may be worth discussing CEMPs since they have enough literature although not necessarily relavent here. Carigi+Peimbert 2011, Molla+2015. Lyman limit systems in Lehner+2016.}


\dbnote{other stellar surveys...
Gustafsson+1999, bensby\&feltzing+2006, Spite+2005, Neiva \& Przybilla+2012, Tautvaisiene+2016.}



\section{Discussion \& Conclusions}\label{sec:conclusions}


Building on~\citet{james+23}, we quantify the impact of C yield assumptions on Milky Way GCE models. We use \citepos{jack} sample of APOGEE subgiants as our primary observational benchmark, as subgiant atmospheres most likely reflect their birth C abundances (see discussion in section~\ref{sec:data_selection}).
In our fiducial model, C initially increases following the slope of the \cc\ C dependence. Later, AGB contributions from C cause a sharp rise in \caah, which slows down due to declining AGB C production and the approach towards equilibrium. The chemical elemental reach a quasi-equilibrium within \about{5}\,Gyr. As a result, the current C/Mg versus metallicity gradient is a superposition of equilibrium states, mixed together from different Galactic positions.


The slope of the predicted \caah\ relation is principally sensitive to the collective metallicity dependence of \cc\ C yields.
While AGB yields are predicted to decline with metallicity, out models show that \cc\ dominate the trend and make an overall positive metallicity trend. 
As the strength of the \cc\ metallicity dependence increases, the slope of the \caah\ trend correspondingly increases. The small effects of \agb\ C yield on the \caah\ trend can be easily corrected by small adjustments of the \cc\ yield's metallicity dependence.
Our predicted slope of NUMBER is approximately consistent with the \citet{LC18} rotating stellar models, however no simulation contains sufficient mass resolution or accuracy to reproduce the observed trends accurately. 

Because massive star enrichment dominates the C mass budget in our models, the predictions are relatively insensitive to the choice of AGB star yield model.
Of the \agb\ tables tested here, \fruity, \monash, and \nugrid\ all predict similar abundance trends in [C/Mg] with [Mg/H] and [Fe/Mg]. The strong destruction of C by massive stars at \about{} solar metallicity   in \aton\ are instead in tension with the data (See Figs.~\ref{fig:agb_predictions})  and~\ref{fig:agb-ssp})
As both the \fruity\ and \nugrid\ models predict \about{} linear N yields as well, these combined models best explain combined C and N abundance trends (see Appendix!?).

As in~\citet{james+23}, we have constrained yield ratios as opposed to absolute yields. For example, scaling yields and outflows by a corresponding factor leaves abundance ratio trends unchanged (Fig.~\ref{fig:sims_degens}).  Effects such as black hole formation could create systematic shifts on predicted chemical yields.
Variations in the SFH instead only induce minor systematic shifts


When combined with~\citeauthor{james+23}'s~\citeyearpar{james+23} empirically calibrated N yields, we find that our fiducial model accurately describes the [C/N]-[Mg/H] relation in our sample. The [C/N]-[Mg/Fe] relation, on the other hand, is shallower than the data, indicating that the DTD of C production may be sharper than our AGB yield tables would suggest (or the N DTD more extended, or both).

Finally, we compare our models againsts a compilation of literature gas-phase and halo-star abundances (see Fig.~\ref{fig:gas_phase}). Our fiducial model fails to explain the lowest values of [C/O] at metallicities of -1 to -2. We briefly consider a modified variation of the CCSNe yield of C which drops to explain these values, which maintains trends consistant with APOGEE. 


Our results demonstrate the power of empirically calibrated stellar yields. In our GCE models, trends in abundance ratios with metallicity are largely determined by trends in yield ratios with metallicity. As a result, the metallicity dependence of the total, population-averaged C yield is tightly constrained by the [C/Mg]-[Mg/H] relation, but the metallicity dependencies of the individual contributions from CCSNe and AGB stars are less precisely determined. Due to the sensitivity of elemental yields to poorly understood processes, such as mass-loss rates and convection, our results provide a useful benchmark for stellar evolution models~\citep[see the discussion in e.g.][]{gil-pons+2022}. With abundance measurements for several million stars provided by upcoming spectroscopic surveys, particularly SDSS-V's Milky Way Mapper program~\citep{sdssv}, constraints on both stellar nucleosynthesis and the assembly history of our Galaxy will become increasingly more powerful.

More C observations across different galactic environments will continue to refine a complete understanding of C production, including the yields of the first stars,  evolution in the bulge, and so on. 



\section*{Acknowledgements}

Software that has contributed to this work included  
\VICE~\citep{JW20, james+21},
\textsc{matplotlib} \citep{matplotlib},
\textsc{scipy} \citep{scipy},
\textsc{IPython} \citep{ipy},
\textsc{pandas} \citep{pandas},
\textsc{numpy} \citep{numpy},
\textsc{astropy} \citep{astropy:2013, astropy:2018, astropy:2022},
and 
\textsc{seaborn} \citep{seaborn}
.
Additionally, we thank \citet{OhioSupercomputerCenter1987} for the use of its facilities for the simulations. 

\apogee\ is part of SDSS-IV \citep{sloan_telescope, apogee_instrumentation, sdss_iv_overview, sdss17, apogee17, aspcap}.

Funding for the Sloan Digital Sky 
Survey IV has been provided by the 
Alfred P. Sloan Foundation, the U.S. 
Department of Energy Office of 
Science, and the Participating 
Institutions. 

SDSS-IV acknowledges support and 
resources from the Center for High 
Performance Computing  at the 
University of Utah. The SDSS 
website is www.sdss4.org.

SDSS-IV is managed by the 
Astrophysical Research Consortium 
for the Participating Institutions 
of the SDSS Collaboration including 
the Brazilian Participation Group, 
the Carnegie Institution for Science, 
Carnegie Mellon University, Center for 
Astrophysics | Harvard \& 
Smithsonian, the Chilean Participation 
Group, the French Participation Group, 
Instituto de Astrof\'isica de 
Canarias, The Johns Hopkins 
University, Kavli Institute for the 
Physics and Mathematics of the 
Universe (IPMU) / University of 
Tokyo, the Korean Participation Group, 
Lawrence Berkeley National Laboratory, 
Leibniz Institut f\"ur Astrophysik 
Potsdam (AIP),  Max-Planck-Institut 
f\"ur Astronomie (MPIA Heidelberg), 
Max-Planck-Institut f\"ur 
Astrophysik (MPA Garching), 
Max-Planck-Institut f\"ur 
Extraterrestrische Physik (MPE), 
National Astronomical Observatories of 
China, New Mexico State University, 
New York University, University of 
Notre Dame, Observat\'ario 
Nacional / MCTI, The Ohio State 
University, Pennsylvania State 
University, Shanghai 
Astronomical Observatory, United 
Kingdom Participation Group, 
Universidad Nacional Aut\'onoma 
de M\'exico, University of Arizona, 
University of Colorado Boulder, 
University of Oxford, University of 
Portsmouth, University of Utah, 
University of Virginia, University 
of Washington, University of 
Wisconsin, Vanderbilt University, 
and Yale University.

\dbnote{do I need acknowledgments for appendix spectro-surveys?}

%%%%%%%%%%%%%%%%%%%%%%%%%%%%%%%%%%%%%%%%%%%%%%%%%%
\section*{Data Availability}

\dbadd{data and codes used in this paper are all publicly available. (do i need links here?). }


%%%%%%%%%%%%%%%%%%%% REFERENCES %%%%%%%%%%%%%%%%%%
\bibliographystyle{mnras}
\bibliography{main}


%%%%%%%%%%%%%%%%% APPENDICES %%%%%%%%%%%%%%%%%%%%%

\appendix


\section{Monte-Carlo Markov Chain inference}

As a way to quickly explore which combinations of yields agree with \caah, and \caafe, we run Monte Carloe Marcov chains using the following approximation framework. We intend these results to be more illustrative and first order, as the number of approximations and systematics in the analysis is substantial. 

\subsection{Methods}

As a first step, we hold all GCE parameters except for the C yield constant for each MCMC model. We additionally assume that the metallicity is that of the fiducial yield set, to avoid second-order effects from metallicity-dependent yields. We then seperate each process into four different components
\begin{enumerate}
    \item $\alpha$ AGB yield (scale).
    \item $\zetao$ Constant CCSNe yield
    \item $\zetai$ Linear CCSNe yield. Held constant past [M/H] = -2
    \item $\zetaii$ quadratic CCSNe yield. Also held constant ...
\end{enumerate}
Since each process is tracked separately, we can then predict the total C abundance at any point in the model by adding together the C abundances from each process
\begin{equation}
    Z_{\rm C} = \sum_{i} \alpha_i Z_{\rm C}^{i}. 
\end{equation}
Since the models do also have statistical uncertainties, the variance of the mean is given by
\begin{equation}
    V(\bar Z_{\rm C}) = \sum_{i} \alpha_i^2 V(\bar Z_{\rm C}^{i})
\end{equation}


To formulate a likelihood, we assume that model agreement is characterized primarily by the \textit{mean trends} in [C/Mg] with [Mg/Fe] and [Mg/H]. For each bin, we calculate the mean $Z_{\rm C}$ for each model component. Then these means are added together and the resulting $Z_{rm C}$ is compared to the $Z_{\rm C}$ of the subgiants in the bin, using a likelihood
\begin{equation} \label{eq:mcmc_likelihood}
    \log {\cal L} \propto \chi^2 = \sum_b \frac{(\bar Z_{b, \rm C,\ model} - \bar Z_{b, \rm C,\ obs})^2} {s_{b, \rm obs}^2 + {\rm Var}(\bar Z_{b, \rm C,\ model})}
\end{equation}

We adopt the following priors
\begin{subequations}
\begin{align}
    \alpha &\sim N(1, 1) \\
    \zeta^{(0)} &\sim N(2, 1) \\
    \zeta^{(1)} &\sim N(0, 1) \\
    \zeta^{(2)} &\sim N(0, 4) \\
\end{align}
\end{subequations}

We then run a Monte-Carlo Markov Chain (MCMC) using Turing.jl with a No-U-turn sampler (acceptance rate of 0.65), combining 8 walkers with 10,000 steps each.

(in progress) To ensure this method is relatively robust, we will perform similar MCMC simulations varying the bin size/method, the MCMC sampler (HMC, ensemble stretch moves from pyemcee, MH, MH random walk), the likelihood (including median \& distribution based), and explore 2D bins. In each case, the derived parameters are within their uncertainties (excluding the KS test model). Finally, we additionally do run multizone simulations for the fiducial model varying each parameter by 5-sigma (from the distribution) and ensuring that the resulting trend is within the uncertainty of this approximation.

(in progress) We check that removing the likelihood in \caafe{} causes...
Additionally, we try and fit models without AGB components, without quadratic components, etc., finding that each model fits the data poorer than the above formulation.

\subsection{Results}

Fig.~\ref{fig:mcmc} displays the distribution and correlation of each parameter in the fiducial MCMC model. Each parameter is relatively well-constrained. The most prominant correlation, between $\zetao$ and $\alpha$, represents that the total solar-yield is well constrained. The other main correlation, between $\zetai$ and $\zetaii$, arises from the shape of the \caah\ trend. If the quadratic component is decreased, than the slope  (past M/H $\sim 0.1$) increases, so the linear component must decrease in order to maintain agreememnt with the data.

Fig.~\ref{fig:mcmc_caahfe} shows the resulting mean trends in \caah\ and \caafe\ for the median best MCMC parameters for each AGB mode. Regardless of the AGB model, there is a reasonable choice of \cc\ yields which can precisely match the \caah\ trend. On the other hand, the agreement with the \caafe\ trend tends to be more poor for each AGB model, where they all predict a later decrease in \caafe\ than is observed. This is improved by yield models with a higher \agb\ enrichment from low mass stars ($\lesssim 2.5$). 

Fig.~\ref{fig:mcmc_ytot} shows the AGB fraction at solar metallicity and the total C yield with metallicity for each MCMC run. 
In essentially each case, the majority of samples have a AGB fraction between 0.15 and 0.4. 

\begin{figure}
\centering
\includegraphics{figures/mcmc_corner.pdf}
\caption[]{
MCMC samples. Could actually just try plotting the distributions from each model here...
}

\label{fig:mcmc}
\end{figure}

\begin{figure*}
    \centering
    \includegraphics[]{figures/mcmc_caahfe_predicted.pdf}
    \caption{Predicted mean tracks for the mean models. \dbnote{This figure is almost identical to Figure 8, so may need to think about if we need both....especially since the MCMC now informs the choice of parameters in Fig. 8.}}
    \label{fig:mcmc_caahfe}
\end{figure*}


\begin{figure*}
    \centering
    \includegraphics[]{figures/mcmc_fagb.pdf}
    \includegraphics[]{figures/mcmc_y_tot.pdf}
    \caption{Caption}
    \label{fig:mcmc_ytot}
\end{figure*}


\section{Model Parameters}

\begin{table*}
    \caption{
    Best-fit parameters from MCMC analysis for various models discussed.
    \dbnote{Not sure if we should use analytic, nugrid, shifted fruity or normal fruity yields as fiducial here.}}
    \begin{tabular}{l l l l l l l l l l} 
    \hline
    model            & $\chi2$  & $\log p$ & alpha & zeta0 & zeta1 & zeta2 & fagba & ytota & zeta1a\\
\hline
analytic         &      5.0 &     0.94 & $0.38^{+0.02}_{-0.02}$  &  $2.41^{+0.02}_{-0.02}$  &  $1.51^{+0.04}_{-0.04}$  &  $2.24^{+0.13}_{-0.13}$  &  $0.14^{+0.01}_{-0.01}$  &  $4.28^{+0.01}_{-0.01}$  &  $1.73^{+0.07}_{-0.07}$\\ 
lateburst        &      8.7 &   -26.31 & $0.48^{+0.03}_{-0.03}$  &  $2.28^{+0.03}_{-0.03}$  &  $1.90^{+0.09}_{-0.09}$  &  $2.95^{+0.22}_{-0.21}$  &  $0.17^{+0.01}_{-0.01}$  &  $4.23^{+0.01}_{-0.01}$  &  $2.18^{+0.13}_{-0.13}$\\ 
twoinfall        &      7.2 &    -5.68 & $0.55^{+0.05}_{-0.04}$  &  $2.36^{+0.04}_{-0.04}$  &  $2.12^{+0.11}_{-0.10}$  &  $1.96^{+0.20}_{-0.19}$  &  $0.19^{+0.02}_{-0.01}$  &  $4.48^{+0.02}_{-0.02}$  &  $2.41^{+0.16}_{-0.15}$\\ 
eta2             &      6.2 &    -1.85 & $0.55^{+0.03}_{-0.03}$  &  $2.19^{+0.03}_{-0.03}$  &  $1.39^{+0.04}_{-0.04}$  &  $2.81^{+0.15}_{-0.15}$  &  $0.20^{+0.01}_{-0.01}$  &  $4.20^{+0.01}_{-0.01}$  &  $1.29^{+0.05}_{-0.05}$\\ 
fruity           &      6.7 &    -8.99 & $2.05^{+0.13}_{-0.13}$  &  $2.00^{+0.05}_{-0.05}$  &  $2.27^{+0.07}_{-0.07}$  &  $2.54^{+0.14}_{-0.14}$  &  $0.28^{+0.02}_{-0.02}$  &  $4.26^{+0.01}_{-0.01}$  &  $2.38^{+0.07}_{-0.07}$\\ 
fruitym0.7      &      5.6 &    -0.62 & $0.99^{+0.06}_{-0.06}$  &  $2.42^{+0.02}_{-0.02}$  &  $1.63^{+0.04}_{-0.05}$  &  $2.38^{+0.14}_{-0.14}$  &  $0.15^{+0.01}_{-0.01}$  &  $4.36^{+0.01}_{-0.01}$  &  $2.03^{+0.07}_{-0.07}$\\ 
aton             &      9.1 &   -19.66 & $1.78^{+0.12}_{-0.12}$  &  $2.49^{+0.02}_{-0.02}$  &  $3.60^{+0.16}_{-0.15}$  &  $2.03^{+0.14}_{-0.15}$  &  $0.09^{+0.01}_{-0.01}$  &  $4.21^{+0.01}_{-0.01}$  &  $2.78^{+0.08}_{-0.08}$\\ 
monash           &     11.1 &   -23.94 & $2.92^{+0.23}_{-0.22}$  &  $1.56^{+0.09}_{-0.10}$  &  $4.74^{+0.26}_{-0.25}$  &  $4.31^{+0.26}_{-0.25}$  &  $0.34^{+0.03}_{-0.03}$  &  $3.65^{+0.05}_{-0.05}$  &  $2.34^{+0.10}_{-0.10}$\\ 
nugrid           &     10.3 &   -14.80 & $0.93^{+0.05}_{-0.05}$  &  $1.71^{+0.06}_{-0.06}$  &  $1.26^{+0.04}_{-0.04}$  &  $2.78^{+0.14}_{-0.14}$  &  $0.31^{+0.02}_{-0.02}$  &  $3.80^{+0.03}_{-0.03}$  &  $1.28^{+0.08}_{-0.08}$\\ 
v21              &     20.0 &  -205.34 & $0.68^{+0.03}_{-0.03}$  &  $3.45^{+0.03}_{-0.03}$  &  $1.84^{+0.10}_{-0.10}$  &  $1.71^{+0.24}_{-0.25}$  &  $0.16^{+0.01}_{-0.01}$  &  $6.33^{+0.02}_{-0.02}$  &  $1.78^{+0.15}_{-0.16}$\\ 
gso              &      3.5 &   -14.96 & $0.95^{+0.51}_{-0.49}$  &  $3.26^{+0.45}_{-0.46}$  &  $-0.66^{+0.90}_{-0.91}$  &  $-1.15^{+2.65}_{-2.70}$  &  $0.23^{+0.12}_{-0.11}$  &  $6.47^{+0.25}_{-0.25}$  &  $-2.46^{+1.67}_{-1.71}$\\ 
galah            &     12.3 &   -85.65 & $0.48^{+0.02}_{-0.02}$  &  $2.80^{+0.02}_{-0.02}$  &  $0.90^{+0.05}_{-0.05}$  &  $2.13^{+0.13}_{-0.13}$  &  $0.15^{+0.01}_{-0.01}$  &  $5.03^{+0.01}_{-0.01}$  &  $0.65^{+0.07}_{-0.07}$\\ 


\hline
total & &   &  $0.65^{+1.25}_{-0.19}$  &  $2.36^{+0.45}_{-0.43}$  &  $1.63^{+0.75}_{-0.42}$  &  $2.29^{+0.62}_{-0.43}$  &  $0.19^{+0.14}_{-0.05}$  &  $4.26^{+2.07}_{-0.07}$  &  $1.68^{+0.74}_{-1.11}$\\
\hline
    \end{tabular}

    
\end{table*}

\begin{table*}
	\centering
    \caption[]{Description of the models presented in this paper.}
	\label{tab:model_parameters}

	\begin{tabular}{l l l l l l l l l l}
		\hline
            name & $\alpha_{\rm C}^{\rm AGB}$ & $Y_{\rm C}^{\rm AGB}$ & $\zeta_0$ & $\zeta_1$ & $\zeta_2$ \\ 
            \hline
            fiducial & 2.04 & fruity & 2.01 & 2.27 & 2.53 \\
            FRUITY & 1 & fruity & 2.39 & 1.86 & 2.37 \\
            Monash & 1 & monash & 2.36 & 2.60 & 2.95 \\
            ATON & 1 & aton & 2.61 & 2.67 & 2.12 \\
            NuGRID & 1 & nugrid & 1.78 & 1.62 & 3.01 \\
            f=0.0 &  \\
            f=0.5 & 3.623 & fruity  \\
            steep & 2.04 & fruity \\
            shallow & 2.04 & fruity \\
            fz=0 \\
            fz=0.5 \\
            m0.5 & 2.04 & fruity(m/0.5)  & 2.44 & 1.42 & 2.11 \\
            m0.7 & 2.04 & fruity(m/0.7) & 2.03 & 1.92 & 3.03 \\
            m1.5 & 2.04 & fruity(m/1.5) & 2.08 & 1.88 & 1.87 \\
		\hline
	\end{tabular}
\end{table*}



For our added scatter, we use the polynomial fits to the reported internal APOGEE error with metallicity. In detail, these polynomials should also very with logg, teff, however as all of our stars are subgiants, these effects should be smaller. We also use $x={\rm [Fe/H]}$ for brevety.
\begin{subequations}
\begin{align}
    \delta {\rm [Mg/H]} &= 0.0652 x^2 + 0.00522 x + 0.0338 \\
    \delta {\rm [Mg/Fe]} &= 0.00793 x^2 - 0.00802 x + 0.0138 \\
    \delta {\rm [C/Mg]} &= -0.0378x + 0.03506 
\end{align}
\end{subequations}

Low-$\alpha$ stars are defined to have 
\begin{equation}\label{eq:high_alpha}
\begin{cases}
\text{[Mg/Fe]} <0.16-0.13\,\text{[Fe/H]}, & \text{[Fe/H]}<0\\
\text{[Mg/Fe]} <0.16, & \text{[Fe/H]}>0. \\
\end{cases}
\end{equation}




\bsp	% typesetting comment
\label{lastpage}
\end{document}




