% mnras_template.tex 
%
% LaTeX template for creating an MNRAS paper
%
% v3.0 released 14 May 2015
% (version numbers match those of mnras.cls)
%
% Copyright (C) Royal Astronomical Society 2015
% Authors:
% Keith T. Smith (Royal Astronomical Society)

% Change log
%
% v3.0 May 2015
%    Renamed to match the new package name
%    Version number matches mnras.cls
%    A few minor tweaks to wording
% v1.0 September 2013
%    Beta testing only - never publicly released
%    First version: a simple (ish) template for creating an MNRAS paper

%%%%%%%%%%%%%%%%%%%%%%%%%%%%%%%%%%%%%%%%%%%%%%%%%%
% Basic setup. Most papers should leave these options alone.
\documentclass[fleqn,
%referee,
usenatbib]{mnras}


% MNRAS is set in Times font. If you don't have this installed (most LaTeX
% installations will be fine) or prefer the old Computer Modern fonts, comment
% out the following line
\usepackage{newtxtext,newtxmath}
\usepackage{anyfontsize}
% Depending on your LaTeX fonts installation, you might get better results with one of these:
%\usepackage{mathptmx}
%\usepackage{txfonts}

% Use vector fonts, so it zooms properly in on-screen viewing software
% Don't change these lines unless you know what you are doing
\usepackage[T1]{fontenc}

% Allow "Thomas van Noord" and "Simon de Laguarde" and alike to be sorted by "N" and "L" etc. in the bibliography.
% Write the name in the bibliography as "\VAN{Noord}{Van}{van} Noord, Thomas"


%%%%% AUTHORS - PLACE YOUR OWN PACKAGES HERE %%%%%

% Only include extra packages if you really need them. Common packages are:
\usepackage{graphicx}	% Including figure files
\usepackage{amsmath}	% Advanced maths commands
% \usepackage{amssymb}	% Extra maths symbols

\usepackage{isotope}

\usepackage{hyperref}

\graphicspath{{figures/}} 




%%%%%%%%%%%%%%%%%%%%%%%%%%%%%%%%%%%%%%%%%%%%%%%%%%

%%%%% AUTHORS - PLACE YOUR OWN COMMANDS HERE %%%%%

% Please keep new commands to a minimum, and use \newcommand not \def to avoid
% overwriting existing commands. Example:
%\newcommand{\pcm}{\,cm$^{-2}$}	% per cm-squared


% citations
\defcitealias{james+21}{J21}
\newcommand{\JJ}{\citetalias{james+21}}
\newcommand{\VICE}{\textsc{vice}}
\newcommand{\citetjack}{Roberts et al.~(2023, in prep.)}
\newcommand{\citepjack}{(Roberts et al.~2023, in prep.)}
\newcommand{\citealtjack}{Robert et al.~2023, in prep.}
\newcommand{\cxi}{\texttt{\hyperlink{C11}{C11}}}
\newcommand{\kx}{\texttt{\hyperlink{K10}{K10}}}
\newcommand{\kxvi}{\texttt{\hyperlink{K16}{K16}}}
\newcommand{\vxiii}{\texttt{\hyperlink{V13}{V13}}}
\newcommand{\cfactor}{2.3}



% Acronyms
\newcommand{\agb}{AGB}
\newcommand{\apogee}{APOGEE}
\newcommand{\cc}{CCSNe}
\newcommand{\gce}{GCE}
\newcommand{\ia}{SNe Ia}
\newcommand{\imf}{IMF}
\newcommand{\sfh}{SFH} % check where first defined
\newcommand{\ssp}{SSP}

% internal abbreviations
\newcommand{\lims}{lower-intermediate mass stars}
\newcommand{\hms}{high-mass stars}
\newcommand{\caah}{[C/Mg]-[Mg/H]}
\newcommand{\caafe}{[C/Mg]-[Mg/Fe]}


% other
\newcommand{\ycmg}{\ensuremath{2.7 + 32\left(Z-\Zo\right)}}
\newcommand{\ycccmg}{1}
\newcommand{\yco}{1}
\newcommand{\fmeas}{20\%}

\makeatletter
\newcommand{\C}[1][\@nil]{
    \def\tmp{#1}%
    \ifx\tmp\@nnil%
        \ensuremath{\rm C}%
    \else%
        \ifmmode ^{#1}{\rm C}%
        \else $^{#1}$C%
        \fi%
\fi }
\makeatother

\newcommand{\Yct}{\ensuremath{y_{\rm C}}}
\newcommand{\Ycc}{\ensuremath{y_{\rm C}^{\rm cc}}}
\newcommand{\Yoc}{\ensuremath{y_{\rm Mg}^{\rm cc}}}
\newcommand{\Ycagb}{\ensuremath{y_{\rm C}^{\rm agb}}}
% \newcommand{\y}{\ensuremath{\rotatebox[origin=B,y=0.5ex]{180}{y}}}
\newcommand{\y}{Y}

\newcommand{\Mo}{%
    \ifmmode {\rm M}_{\sun}%
    \else {M$_{\sun}$}%
    \fi}
\newcommand{\Zo}{%
    \ifmmode Z_{\sun}%
    \else $Z_{\sun}$%
    \fi}

\newcommand{\about}[1]{${\sim} #1$}

%%%%%%%%%%%%%%%%%%%%%%%%%%%%%%%%%%%%%%%%%%%%%%%%%%

%%%%%%%%%%%%%%%%%%% TITLE PAGE %%%%%%%%%%%%%%%%%%%

% Title of the paper, and the short title which is used in the headers.
% Keep the title short and informative.
\title[The Origin and Galactic Evolution of Carbon]{The Galactic Chemical Evolution of Carbon: Implications for Stellar Nucleosynthesis }

% The list of authors, and the short list which is used in the headers.
% If you need two or more lines of authors, add an extra line using \newauthor
\author[D. A. Boyea et. al.]{
Daniel A. Boyea,$^{1}$\thanks{E-mail: boyea.2@osu.edu}
James W. Johnson,$^{1}$
Third Author$^{2,3}$
and Others$^{1,3}$
\\
% List of institutions
$^{1}$Department of Astronomy, the Ohio State University, 191 W. Woodruff, Columbus, OH 43210, USA
}

% These dates will be filled out by the publisher
\date{Accepted XXX. Received YYY; in original form ZZZ}

% Enter the current year, for the copyright statements etc.
\pubyear{2023}

% Don't change these lines
\begin{document}
\label{firstpage}
\pagerange{\pageref{firstpage}--\pageref{lastpage}}
\maketitle



% Abstract of the paper
\begin{abstract}
% context
C is an important element across astronomy; however, its origin remains poorly understood. 
% 
We aim to constrain the stellar yields of C through multi-zone Galactic chemical evolution models by comparing predictions with \apogee\ subgiants abundances.
% 
We find that \caafe\ is an emirical estimate of the delayed C sources, enabling us to estimate that \agb\ stars and \cc\ produce about 20\% and 80\% of C, respectively.  
The \caah\ trend instead represents the equilibrium abundances of C and Mg. 
We use the \caah\ trend to estimate the CCSNe C/Mg yield, determining that  $\Ycc/\Yoc = EQUATION$ when including \agb\ C. 
% misc
Our models are relatively independent of uniform scaling of yields and outflows, and alternate star formation histories. 
However, the stars which contribute to \agb\ C production and the \ia\ delay time distribution of Fe contribute uncertainties to our conclusions. 
% gas phase
While reliable gas-phase and low-metallicity measurments of C are challenging, we find that our model and a singlezone model with our recommended yields replicate the broad trends of \caah{} across different environments and metallicities. 

\end{abstract}





%%%%%%%%%%%%%%%%%%%%%%%%%%%%%%%%%%%%%%%%%%%%%%%%%%

%%%%%%%%%%%%%%%%% BODY OF PAPER %%%%%%%%%%%%%%%%%%

\section{Introduction}


Carbon is a distinctive and well-studied element in astronomy. 
Formed in the cores of stars during He fusion, C is the lightest directly synthesized element after He, one of the only light elements formed in low-mass stars, and one the most abundant metals \citep[e.g.][]{jennifer19, KL14}.\footnotemark{}
C structurally changes the environments it pollutes -- regulating stellar evolution, facilitating the formation of stars and planets, and forming the basis of earthly life.
% Because the effects of first dredge up are mass-dependent, C and N abundances are frequently used as age indicators for RGB stars \citep{MG15, martig16, hasselquist19, vincenzo+21}. 
As such, understanding the origin of C has wide ranging implications. 
While we know both lower-intermediate-mass and high-mass stars produce C, the relative importance of each process is still unknown.

\footnotetext{By metallicity, we mean the (mass) fraction of any element which is not H or He, denoted by $Z$. For the sun, we take $Z_\odot=0.014$. }



We know, from observations of MW stars and (extragalactic) gas, that C/O traces a banana shape in O/H (see Fig.~\ref{fig:gas_phase} and section~\ref{sec:gas}). 
At the very lowest metallicities, where $[{\rm O/H}]\lesssim-2$,%
\footnote{In this paper, we use the standard notation for chemical abundances. $[A/B] = \log_{10}\left(A/B\right) - \log_{10}\left(A_{\sun}/B_{\sun}\right)$, i.e. $[A/B]$ is the logarithm of the ratio between A and B, scaled such that $[A/B]=0$ for the sun. Solar abundances are as measured in \citet{asplund+09}.}
%
observations suggest that C/O declines with metallicity.
From $-2 \lesssim {\rm [O/H]}\lesssim -1$, C/O is roughtly constant with metallicities. 
And, at higher metallicities $-1 \lesssim {\rm [O/H]}$, C/O increases with increasing metallicity.
    As our primary observational constraint, we use a sample of subgiant stars from the \apogee{} \citep{apogee17} selected by the criteria in Roberts et al. (2023, in prep).
    According to stellar evolution theory and observations \citep{gilroy89, korn+07, lind+08, souto+18, souto19}, these stars have not yet experienced \textit{first dredge up}%
\footnote{First dredge up is when material from CNO processed core is mixed with the (convective) stellar envelop during the ascent onto the RGB. As a result, C is decreased and N is increased depending on the strength of this process and the envelop mass.}
yet have well-mixed envelopes. So, subgiant surface abundances most accurately represent their birth composition. 
In contrast, evolved RGB stars require modelling-dependent corrections to recover birth C abundances \citep[e.g.][]{vincenzo+21}. 
Fig.~\ref{fig:subgiants} shows the subgiant sample plotted in \caah\ and \caafe{}. [C/Mg] increases with metallicity, and [C/Mg] decreases with [Mg/Fe] at fixed [Mg/H]. 
Using the subgiant abundance trends, we will develop a model of the enrichment sources and evolution of C.







\begin{figure*}
    \centering
    \includegraphics{subgiants.pdf}
    \caption{The [C/Mg] ratio against [Mg/H] (top) and [Mg/Fe] (bottom) for the \citetjack~sample of \apogee{} subgiants. On the top, we plot high and low-$\alpha$ stars in blue and orange, using the separation defined in Equation \ref{eq:high_alpha} (the high and low-$\alpha$ stars are named for their high or low $\alpha$-element to Fe ratios, or in this case, Mg/Fe). On the bottom, we colour-code stars according to their [Mg/H] abundance.} \label{fig:subgiants}
\end{figure*}



Galactic chemical evolution (\gce) is a powerful tool, capable of uncovering the origins of the elements. 
Each enrichment process has characteristic chemical signatures and timescales, enabling us to reconstruct chemical histories.
Many previous works have used GCE in attemt to understand C abundances, whether through using theoretical stellar models \citep{DTS78, prantzos+18, chiappini+03} 
or understanding  observation (\citealt{tinsley79, HEK00, BF06, rybizki+17, berg+19, KKL20};
See also review in \citealt{romano22}).
%
Every study agrees that C is produced by a combination of high-mass and \lims{}, different studies disagree on which process is dominant. 
For example CITES conclude \hms{} dominate wherease CITES conclude \lims{} contribute the majority of C.
C is also generally understood to have strong metallicity dependent CCSNe enrichment. 


One of the primary uncertainties of GCE models are nucleosynthetic yields. Yield predictions -- the amounts of each chemical element stars produce --
are shaped by poorly understood processes, including mass loss, nuclear reaction rates, rotational mixing, convection, and explodability \citep{romano+10,KL14,ventura+13, LC18, emily+21}.
To better understand where C comes from and how it evolves, our aim is to combine \apogee\ observations and multi-zone models to develop observationally-consistant yields.
\cite{james+23} examined similar \gce{} models of N (which is closely related to C), finding that trends in N and O are explained by the metallicity dependence of N/O yields. \citet{james+23} determine that \agb\ N abundances roughly depend linearly on metallicity (i.e. $y_{\rm N}/y_{\rm O} \propto Z$). 
Here, we extend their models to C, deriving similar constraints on C/Mg yields. We assess which yield prescriptions reproduce Galactic abundance trends while investigating the impact of \gce{} model assumptions, such as the star formation history (\sfh{}) and outflow mass loading.







\section{Nucleosynthesis}
We adapt the yield choices of elements besides C from \citet{james+21, james+23}.
Table \ref{tab:fiducial_mod} contains our fiducial yields. 
Following \citet{james+21, james+23}, we also take the \ia{} delay time distribution to be a $t^{-1.1}$ power-law with a minimum delay time of 140\,Myr, as suggested by the observations of \citet{maoz+12}.


Yields -- the amounts of each chemical element stars synthesise -- are central to studies of galactic chemical evolution (GCE). 
In this section, we compare literature yield predictions and discuss our fiducial yield choices. 
We focus on three nucleosynthetic pathways: asymptotic giant branch (\agb{}) stars, core collapse supernovae (\cc{}), and type Ia supernovae (\ia{}).
C is produced in both \agb{} and \cc{} stars.
We also use Mg and Fe as tracers of \cc{} and \ia{} enrichment respectively. 
O and Mg are produced almost entirely from \cc\ with metallicity-independent yields. In contrast, Fe is produced in similar amounts by \cc\ and \ia.

After a single stellar population (SSP)\footnotemark{} forms, \cc{} are the first chemical enrichers. \cc{} explode within $\lesssim 40$\,Myr, providing light elements (e.g. C, O, and Mg) and heavier elements (Fe and beyond). Next, low-mass stars begin to reach the end of their lives, entering the \agb{} phase. By shedding their outer layers, \agb{} stars are important sources of C, N, and neutron capture elements.  Finally, white dwarfs explode in \ia{}, releasing Fe and other iron-peak elements.

\footnotetext{SSP: Single stellar population. Technical name for a group of stars born in the same conditions at the same time, i.e. an open cluster.}


To quantify yields, we define the stellar yield to be the fraction of a stars initial mass which is newly synthesized and released as a given element. 
For an element $X$ and star with mass $M$, the net-fractional stellar yield $\y$ is 
\begin{equation}
\y_{X} =  \Delta Z_X \frac{M_{\rm ejected}}{M_{\rm birth}}
\end{equation}
where $M_{\rm ejected}$ and $M_{\rm birth}$  are the total ejected mass and the birth mass of the star, and $\Delta Z_X$ is the change in $Z_X$ from the birth material to the ejected material of the star.%
\footnote{$Z_{X}$ represents the mass fraction of element $X$.}
%
For example, a 1\,\Mo\ star with $\y_{\C} = 10^{-3}$ will add $10^{-3}\,\Mo$ of new C to the interstellar medium. 
Also, note that yields may be negative if the material returned to the interstellar medium has a lower abundance $Z_X$ than the material the star was formed from.
Although per-star yields are necessary to compute \agb{} star enrichment rates in \gce{}  models, initial-mass-function (\imf) averaged yields are useful in interpreting their predictions. An IMF-averaged yield adds together the yields of stars of each mass, weighted by the fraction of stars of each mass (the \imf). 
Given the fractional yield $\y(M, Z)$ as a function of initial stellar mass $M$ and metallicity $Z$, the \imf-averaged yield is given by 
\begin{equation} \label{eq:imf-yield}
    y_{\rm X}(Z,t) = 
    \frac{
    \int_{M_{\rm min}(t)}^{M_{\rm max}} 
    \y_{\rm X}(M, Z)
    \frac{dN}{dM}\ M\ dM
}
{
    \int \frac{dN}{dM}\ M\ dM
}
\end{equation}
where ${dN}/{dM}$ is the \imf, $M_{\rm max}=100\,\Mo$ is the maximum stellar mass, and $M_{\rm min}(t)$ is the mass of stars with lifetime $t$.\footnotemark{}
To calculate the \imf-averaged net yields, we use the Versatile Integrator for Chemical Evolution code (\VICE).\footnotemark{} 


\addtocounter{footnote}{-2}
    \stepcounter{footnote}\footnotetext{In our model, the mass-lifetime relation is
$\log \tau_M = 1.02 - 3.57\log M + 0.90 \left(\log M\right)^2$,
where $\tau_M$ is in Gyr, from \citealt{larson74}.
We use $t_{\rm end}=10\,$Gyr for total yields when $t$ is not used.}

\stepcounter{footnote}\footnotetext{\VICE~is available at \url{https://github.com/giganano/VICE}}


\begin{table}
	\centering
    \caption[]{Yields for the fiducial model (in units of \ssp~birth mass). See Section \ref{sec:agb} for the definition of \cxi.}
	\label{tab:fiducial_mod}

	\begin{tabular}{l l l l}
		\hline
        Element & $y^{\rm cc}$ & $\y^{\rm agb}$ & $y^{\rm ia}$ \\
		\hline
        C & Eq.~\ref{eq:zeta} & $2.3\times$\cxi &  0 \\
        O & 0.015 & 0 & 0 \\
        Mg & 0.00185 & 0 & 0 \\
        Fe & 0.0012 & 0 & 0.00214 \\
        N & 0.00072 & 0.0009$M\left(\frac{Z}{Z_\odot}\right)$ & 0\\
		\hline
	\end{tabular}
\end{table}

\subsection{Asymptotic Giant Branch Stars}\label{sec:agb}

An \agb\ star is a low-mass ($\lesssim 8\,\Mo$) star during its final phase of evolution. Inside an \agb\ star, an inert CO core is surrounded by thermally-unstable He and H-burning shells below a convective envelope. 
In an \agb\ star, two competing processes determine the outcome of C production: \textit{third dredge up} and \textit{hot bottom burning}.  
Third dredge up accompanies thermal pulses in \agb\ stars, where material from the CO core is mixed with the envelope, increasing surface C abundances \citep{KL14}. The C yields of the star are increased as this C-enhanced envelope is released to the interstellar medium. 
Hot bottom burning is the activation of the CNO cycle\footnotemark{}
at the bottom of the convective envelope when $T\gtrsim 50\,{\rm MK}$. Because the $^{14}$N proton capture is the slowest component of the CNO cycle, the CNO cycle converts nearly all \C[12] into $^{14}$N \citep{solar-fusion}.

\footnotetext{
    The CNO cycle is a series of proton-capture reactions with CNO elements resulting in energy generation and the creation of an $\alpha$ particle. $\C[12]({\rm p}, \gamma)
    ^{13}{\rm N}(\beta^+, \nu_{\rm e})
    ^{13}{\rm C}({\rm p}, \gamma)\allowbreak
    ^{14}{\rm N}({\rm p}, \gamma)\allowbreak
    ^{15}{\rm O}(\beta^+, \nu_{\rm e})\allowbreak
    ^{15}{\rm N}({\rm p}, \alpha)
    \C[12]$. 
There are other less important minor branches of the CNO cycle
 \citep{solar-fusion}.
}


Hot bottom burning and third dredge-up result in mass-dependent C yields. 
Stars less than \about{2}\,\Mo do not experience third dredge-up. As a result, these stars C abundances are only affected by first dredge-up, resulting in little change to C yields or slight destruction of C.
Above \about{2}\,\Mo{}, third dredge up becomes important, enriching the outer layers with C.
\agb\ stars between 2 and 5 \Mo are the most abundant producers of C.
In \agb\ stars more massive than \about{5}\,\Mo, both hot bottom burning and third dredge up occur; however, hot bottom burning is much more efficient, resulting in significant \C[12] destruction.


    In this work, we explore four different sets of \agb\ star yield tables from literature,
    providing necessary well-sampled grids in mass and metallicity. We refer to the yields from the following studies as the following,
For our models to match observations, we find that need to uniformly amplify these yield tables. We use \cxi\ table, amplified by a factor of 2.9, as the fiducial \agb\ yield.
Variations in models are due to different treatments of reaction rates, convection, and mass-loss.
\begin{description}
    \item \hypertarget{C11}{\texttt{C11}}: \citet{cristallo+11, cristallo+15}
    \item \hypertarget{K10}{\texttt{K10}}: \citet{karakas10}
\item \hypertarget{V13}{\texttt{V13}}: \citet{ventura+13,ventura+14,ventura+18, ventura+20}
    \item \hypertarget{K16}{\texttt{K16}}: \citet{KL16, karakas+18}
\end{description}



Fig.~\ref{fig:y_agb} compares the stellar \agb\ C yields for these four models.
Most models agree on the qualitative shape of the net fractional \agb\ C yield.
Stellar yields peak between masses of about 2--4 \Mo and decline as stars become more or less massive. As metallicity increases, the total \agb\ C yield decreases. The mass of peak C yields also increases slightly with metallicity. Metal poor stars dredge up more material due to the decreased power of the CNO cycle, resulting in increasing carbon yields with decreasing metalicity \citep{ventura+13}.

Fig.~\ref{fig:agb-ssp}, on the left, shows the total production of C by \agb\ stars in a \ssp{} at an age $t$, i.e. $\y_{\rm C}(Z_\odot, t)$. 
As the mass range $2\,\Mo\lesssim M \lesssim 4\,\Mo$ is most important for C production, about half of C production occurs before \about{1}\,Gyr, similar to \ia\ Fe. 
\kx{} and \kxvi{} weight C production more heavily towards high-mass \agb\ stars resulting in a faster enrichment delay time, whereas the \cxi\ and \vxiii\ models predict a slightly longer timescale of \about{1}\,Gyr. In any case, little to no C is produced more than 2\,Gyr after a star formation event. Fe production, in contrast, continues steadily for 10\,Gyr. 

The right panel of Fig.~\ref{fig:agb-ssp} shows \imf-averaged C yields for each \agb\ model as a function of metallicity.
\vxiii{} differs in that it shows a non-monotonic metallicity dependence. However, this effect is only for models with $\log Z/Z_\odot \lesssim -1$.
Otherwise, models differ only in their yield normalization and metallicity dependence. 
All yield models span a range of \about{2} for a given metallicity.
For example, all four models predict $y_\text{C}^\text{agb}(\Zo)$ to be between 0.004 and 0.008 at solar metallicity. The models also differ slightly in the strength of metallicity dependence, with about a factor of 3 discrepency (see Table~\ref{tab:alpha_agb}).

\begin{table*}
	\centering
    \caption[]{For each \agb\ yield set, the \imf-averaged \agb\ C yield at solar metallicity $y_{\rm C, 0}^{\rm agb}$ and the multiplicative factor reaches an \agb\ contribution of 20\% $\alpha_{\rm agb, 20}$.}

	\label{tab:alpha_agb}
    \begin{tabular}{c ccc p{5cm} p{5cm}} % four columns, alignment for each
		\hline 
        \agb\ table 
                & $y_{\rm C}^{\rm agb}(\Zo)$ & $\zeta^{\rm agb}(\Zo)$ & $\alpha_{20}^{\rm agb}$
                & masses (\Mo) & metallicites 
                \\
        \hline
        \cxi 
                &  0.00042 & -0.0175 & 2.4
                & 1.3, 1.5, 2, 2.5, 3, 4, 5, 6
                & 0.0001, 0.0003, 0.001, 0.002, 0.003, 0.006, 0.008, 0.01, 0.014, 0.02
                \\
        \kx 
                & 0.00064 & -0.059 & 1.6
                & 1, 1.25, 1.5, 1.75, 1.9, 2.25, 2.5, 3, 3.5, 4, 4.5, 5, 5.5, 6 
                &  0.0001, 0.004, 0.008, 0.02
                \\
        \vxiii 
                & 0.00022 & -0.021 & 4.5
                & 1.5, 2, 2.5, 3, 3.5, 4, 4.5, 5, 6, 6.5, 7
                & 0.0003, 0.001, 0.002, 0.004, 0.008, 0.014, 0.04
                \\
        \kxvi 
                & 0.0005 & -0.029 & 2.0
                & 1, 1.25, 1.5, 1.75, 2.25, 2.5, 2.75, 3, 3.25, 3.5, 3.75, 4, 4.5, 5, 5.5, 6, 7 
                & 0.0003, 0.001, 0.002, 0.004, 0.008, 0.014, 0.04
                \\
		\hline
	\end{tabular}
\end{table*}


\begin{figure*}
    \centering
 	    \includegraphics[scale=1]{agb_yields.pdf}
        \caption[]{The net fractional \agb\ C yield  plotted as a function of initial stellar mass $M$ and colour-coded according to metallicity. The black dashed line shows $\y=0$ for reference. Each panel represents yields from one of four \agb\ models: \cxi{}, \kx{}, \vxiii{}, \kxvi{}, and our analytic model (see sections \ref{sec:agb}) }

        \label{fig:y_agb}
\end{figure*}

\begin{figure*}
    \centering
    \includegraphics{y_agb_vs_z.pdf}
    \includegraphics{y_agb_vs_t.pdf}

    \caption[]{ C yields from \agb\ stars as a function of \ssp{} age, as a fraction of the C yield at $t_{\rm end}=10\,$Gyr. 
\textbf{Left} The (\imf-weighted) \agb\ C yield $\Ycagb$ as a function of metallicity for each of the \agb\ yield models. ($\Ycagb$ is the net mass of C produced by \agb\ stars per unit mass of star formation, after 10\,Gyr and assuming a \citealt{kroupa01} \imf.)
    \textbf{Right} Our four considered \agb\ yield models at solar metallicity (\cxi{}, \kx{}, \vxiii{}, \kxvi{}. The dashed red line shows the delay time distribution of type Ia supernovae ($\propto t^{-1.1}$) for comparison, and the minimum of V13 is $y(t)/y(t_{\rm end}=-3.3$. 
}

    \label{fig:agb-ssp}

\end{figure*}



\subsection{Core Collapse Supernovae}


Massive stars form $^{12}$C in their cores through the triple--$\alpha$ process.\footnote{The 3-$\alpha$ process fuses 3 He nuclei into \C[12].} 
However, only C ejected through supernovae and stellar winds contributes to the yield. 
While there are many stellar models providing predictions of \cc{} yields, the results of these models are highly uncertain due to the complexity of stellar modeling.

Fig.~\ref{fig:y_cc} plots calculations of the \imf-integrated yields, defined with Eq.~\ref{eq:imf-yield} (computed using \VICE's \texttt{vice.yields.ccsne.fractional} function). 
\cc{} models predict a wide range of C yields, spanning almost a factor of ten. 
Both the \citet{NKT13} and \cite{LC18} models show positive metallicity dependence. 
As metallicity increases, stars lose more of their mass to winds. In particular, C enriched envelop material is lost through winds before synthesized into heavier elements, so C yields can be strongly metallicity dependent (VERIFY).
Fig.~\ref{fig:y_cc} shows the \cxi{} \agb\ model for comparison on the left. Especially at $Z\approx Z_\odot$, most \cc{} models dominate \agb\ C production. Later, we will also show empirically this is the case.
The right of Fig.~\ref{fig:y_cc} shows the \cc{} [C/Mg] ratio for the different models, defined by
\begin{equation}\label{eq:c_mg_cc}
    {\rm [C/Mg]^{CC}} = \log_{10}\left( \frac{\Ycc}{\Yoc}\right) - \log_{10} \left( \frac{Z_{{\rm C},\ \sun }}{Z_{{\rm Mg},\ \sun }} \right).
\end{equation}
If only \cc{} produced C, then ${\rm [C/Mg]^{CC}}$ describes the equilibrium abundance of [C/Mg].
Different \cc\ models also span a large range in [C/Mg]. 
We chose to instead parameterize $\Ycc$ to simplify model inputs, as most \cc\ models fail to achieve near-solar [C/Mg].

Both rotation and explodability introduce substantial variations. The \cite{LC18} models include rotation, showing that variations in the rotational velocity of the star can dramatically increase the magnitude and metallicity dependence of $\Ycc$. Rotation induces more mixing allowing the CO core to grow larger and contributes to wind losses. As we will later show, \cc\ C production needs to be strongly metallicity-dependent at $Z/Z_\odot \approx 1$, which is consistent with the \cite{LC18} rapidly rotating models.
Assumptions about the explodability landscape affect C and Mg production. As fewer high-mass stars explode, both C and Mg yields decrease, but Mg yields decrease more as more C is lost to winds, so [C/Mg] increases with decreasing explodability. 


O and Mg are both $\alpha$-elements, light elements produced in \cc{} through He-nuclei fusion. 
As we focus on constraining relative yields, we neglect O and Mg yield variations in the main text (excluding the uniform scaling of yields and massloading in Section~\ref{sec:outflows}). There is substantial variation in predicted Mg yields. 
Most models predict relatively flat trends metallicity (even with rotation as in \citealt{LC18}). 
However, the variation is significant and our adopted $\Yoc$ yield is much higher than most models, but O and Mg yields of \cc\ models do not fully match observations. \cc\ models underpredict $[{\rm Mg}/{\rm O}]$, and the reason why is unknown (see e.g. \citealt{emily+21}). Here, we assume ${\rm [O/Mg]} = 0$, which nevertheless consistent with \apogee{} observations \citep{weinberg+19, weinberg+22}.
    

\begin{figure*}
    \centering
    \includegraphics{cc_yields.pdf}
    \caption[]{
        C yields from high-mass stars.
        \textbf{Left} The \imf-weighted \cc\ yield of C as a function of metallicity.
        \textbf{Right} The \cc\ [C/Mg] abundance ratio, defined in Eq.~\ref{eq:c_mg_cc}. The black line is the derived C yield from Section \ref{sec:equilibrium} and Eq.~\ref{eq:zeta}. Yields are shown for tables from 
    \citet[red triangles]{WW95}, \citet[orange squares and diamonds]{sukhbold+16}, 
    \citet[green stars]{NKT13}, and \citet[blue circles]{LC18}. \citet{sukhbold+16} report yields for different black hole landscapes, while \citet{LC18} provide yields at different rotational velocities.
    In the top panel, the pink line denotes $\Ycagb$ from \cxi{} for comparison. All models include wind yields. 
}
    \label{fig:y_cc}
\end{figure*}



\subsection{Equilibrium Abundances}\label{sec:equilibrium}

Here, we assume a simple \textit{one-zone} chemical evolution model \cite[e.g.][]{tinsley80, pagel09, matteucci21}.
Galaxies, when moderated by metal-poor gas accretion and feedback-driven outflows, reach a chemical equilibrium. The production of new metals is balanced by losses to new stars and outflows \citep{larson72, dalcanton07, FD08, PS11, lilly13}.
While our galaxy is likely not in perfect equilibrium or described by a single, homogeneous chemical
envirnoment, the equilibrium approximation is nevertheless useful in understanding yields and
metallicity dependence of solar neighborhood stars \citep[e.g.][]{james_dwarf,james+23,WAF17}. 


\cite{james+23} show that trends in N and O abundance ratios are set by the yield ratio and their metallicity dependences. 
Here, we find a similar conclusion for C and Mg, where equilibrium C/Mg is given by
\begin{equation}\label{eq:equilibrium_yields}
    \frac{Z_{\rm C}^{\rm eq}}{Z_{\rm Mg}^{\rm eq}} = \frac{\Ycc + \Ycagb }{y_{\rm Mg}}.
\end{equation}

In Fig.~\ref{fig:analytic}, we show the inferred total C yields, based on this equation, and our best fitting linear model. From a linear regression, we suggest that
\begin{subequations}\label{eq:yc_inferred}
    \begin{align}
        \frac{\Yct(Z)}{y_{\rm O}} &\approx \frac{1}{3} + 4 \left(Z-\Zo\right) \\
        \frac{\Yct(Z)}{y_{\rm Mg}} &\approx 2.7 + 32 \left(Z-\Zo\right).
    \end{align}
\end{subequations}
These yield ratios results in an equilibrium abundance $[{\rm C}/\alpha] = -0.09$, which is consistent with the subgiant sample and is within \about{20\%} of the solar C/Mg mixture from \citet{asplund+09}.

Both observational and theoretical uncertainties limit the accuracy of our relative yield predictions.
As the metallicity range of our data is small ($-0.4\lesssim {\rm [Mg/H]} \lesssim 0.4$),
our model does not extend beyond near-solar metallicities.
Additionally, the derived yields will be systematically biased if the galaxy is out of equilibrium, for example, due to a recent starburst \citep{mor+19,isern19}. 
As we will discuss in Section \ref{sec:outflows}, the normalization of yields and $\eta$ is degenerate. If we increase yields and $\eta$ by a similar factor, equilibrium abundances from Eq.~\ref{eq:equilibrium_yields} would be unchanged, so our model cannot distinguish changes in both yields and outflows.


With the total C yield of Eq.~\ref{eq:yc_inferred} and given an \agb\ C yield, we can derive an observationally-consistant \cc\ C yield.
\begin{subequations}
    \begin{gather}
        \Ycc(\Zo) = \Yct(\Zo) - \Ycagb(\Zo)\\
        \zeta^{\rm cc} \equiv \frac{d\Ycc(\Zo)}{dZ} = \frac{d\Yct(\Zo)}{dZ} - \frac{d\Ycagb(\Zo)}{dZ}
    \end{gather}
\end{subequations}
This method reduces the number of free parameters of the model and enables all models to match \caah\ trends in \apogee{} subgiants when considering \agb\ yields with different metallicity dependences and normalizations. 
Table~\ref{tab:alpha_agb} contains our adaptive (regression-derived) values of $\zeta^{\rm agb}$ and $\Ycagb(\Zo)$. 
For example, in our fiducial model, $\Ycc = 0.004 + 0.102 (Z-\Zo)$.
Our full \cc\ C yield model adds a low-metallicity enhancement term, which is insignificant at solar metallicities (e.g. thin disk stars) but enables models in Section~\ref{sec:gas} to match low-metallicity environments.
We parameterize low-metallicity enrichment with $y_l$ and $Z_l$, the yield and transition metallicity for enhanced low-metallicity yields.
All-together, 
\begin{equation}\label{eq:zeta}
    \Ycc = y_{\rm C, 0}^{\rm cc} + \zeta^{\rm cc}\,(Z-\Zo) + \frac{2\,y_l}{1 + Z/Z_l}.
\end{equation}



\begin{figure}
    \centering
    \includegraphics{analytic.pdf}
    \caption[]{Inferred total C yields as a function of metallicity. We assume chemical equilibrium (orange curve, see discussion in Section \ref{sec:equilibrium}). Blue points are the median value of $\Ycc$ for each (number) bin in [Mg/H] with uncertainties based on the 16--84 percentile range.
    }
    \label{fig:analytic}
\end{figure}






\section{The Multi-zone Model}\label{sec:vice}

Our models extends the \citet[hereafter \JJ]{james+21} Milky Way model, run with the publicly available Versatile Integrator for Chemical Evolution (\VICE). 
This model is described extensively in \JJ~and concisely summarized  in \citet{james+23}. Here, we provide a brief overview of the relevant model components.
Classical, \textit{one-zone} models of chemical evolution assume instantaneous mixing of metals in the star-forming interstellar medium \citep[e.g.][]{matteucci21}. This simple framework is a poor approximation of the Milky Way.  The Galaxy evolves \textit{inside-out} -- where star formation is higher towards the center and in the early universe \citep{bird+13}. Additionally, stars can migrate several kpc over their lifetimes, mixing different chemical environments across the galaxy \citep{bird+12,sellwood+binney02}. For the rest of this paper, we focus on multi-zone models, which discretize the Galaxy into concentric rings in which stars move between.  

Star formation is set seperately for each zone. The Galaxy is divided into 200 rings, each 100\,pc wide representing a single zone. Each ring (or zone) has a separate stellar population and gas supply. We initially assume an inside-out \sfh{}, where the star formation surface density $\Sigma_\star$ is given by 
\begin{equation}\label{eq:inside_out}
    \dot{\Sigma}_\star \propto \left(1-e^{-t/\tau_{\rm rise}}\right) e^{-t/\tau_{\rm sfh}}.
\end{equation}
$\tau_\text{rise}=2$\,Gyr describes when the star formation rate reaches a maximum, and $\tau_{\rm sfh}$ describes the decay timescale of star formation as a function of radius $R$. \JJ~derives $\tau_{\rm sfh}(R)$ through analysis of four integral field spectroscopy surveys in \cite{sanches20}. At each $R$, the \sfh{} is normalized to match the stellar surface density gradient \citep{BHG16} and the total stellar mass reaches $5.17\times10^{10}\,\Mo$ \citep{LM15}. Star formation ends beyond a radius $R=15.5\,$kpc. 
The gas inflow is calculated to maintain the \sfh{} for each radius and time, using an extension of a Kennicutt-Schmidt law \citep{kennicutt98},
\begin{equation}
\dot{\Sigma}_{\star} \propto 
\begin{cases}
    {\Sigma}_{\rm gas} & 2\times 10^7 \leq \Sigma_{\rm gas} \\ 
    {(\Sigma}_{\rm gas})^{3.6} & 5\times 10^6 \leq \Sigma_{\rm gas} < 2\times10^7 \\ 
    {(\Sigma}_{\rm gas})^{1.7} & \Sigma_{\rm gas} < 5\times10^6 \\ 
\end{cases}
\end{equation} 
where $\Sigma_{\rm gas}$ is measured in \Mo\,kpc$^{-2}$. 
The scaling of this relationship varies with time due to the redshift dependence of $\tau_\star$ in molecular gas observed by \citet{tacconi18}. We assume a \citet{kroupa01} \imf.


To account for radial migration, we use the \texttt{h277} hydrodynamical
simulation results (with simulation parameters as in \citealt{bird+21}; see also \citealt{christensen12, zolotov12, loebman12, BZ14}).  
Each \VICE\ single stellar population (SSP) is matched to an \textit{analog} in \texttt{H277}, chosen to form at a similar time and radius $R$. By taking the change in radius $\Delta R$ of the analogs, the SSPs move to their final radii with a $\sqrt{\text{time}}$ dependence.
The $\Delta R \propto \sqrt{\rm time}$ dependence arises when migration proceeds as a consequence of the diffusion of angular momentum \citep{frankel18, frankel20}.
We do not account for radial gas flows.
Using the results of a hydrodynamical simulation without modification limits the free parameters in the model; however, we are limited to one dynamical history. 
We also explore a normal-distribution random walk migration based on \citet{frankel18}, without noticeable impacts on our results. All models shown here use the \texttt{h277}-based migration. The full impact of the details of a galaxy's dynamical history on its chemical evolution is still unknown.

As the strength of outflows controls the resulting $\alpha$-element abundances, \JJ~create a metallicity gradient by defining
\begin{equation}\label{eq:eta_r}
\eta(R) = r - 1 + \frac{y_{\alpha}^{\rm CC}}{Z_{\alpha, \odot}} 10^{(-0.08\,\text{kpc}^{-1})(R-4\,\text{kpc})+0.3}.
\end{equation}
This choice of $\eta(R)$ results in a [$\alpha$/H] gradient consistent with Milky Way observations \citep[e.g.][]{hayden+14, weinberg+19, frinchaboy+13}.


\section{Data Selection}


Subgiants provide the ideal observational constraint to our model. 
When a star enters the Red Giant Branch (RGB), material from the CNO-processed core is mixed with the envelope in first dredge up, enhancing N and depleting C \citep{iben67, vincenzo+21,KL14}. RGB stars thus require model-dependent corrections to recover surface abundances \citep[e.g.][]{vincenzo+21}. On the other hand, gravitational settling can affect main sequence stellar abundances \citep[e.g.][]{souto19}. Subgiants have well-mixed envelopes, so gravitational settling is not as significant, and subgiants have not yet experienced first dredge up. We use a sample of \apogee\ DR17 stars \citep{apogee17} as selected in \citetjack.  

When plotting the data and models, we bin by groups of 500 (or 150) stars in [Mg/H] ([Mg/Fe]). When plotting \caah, we select only the low-$\alpha$ sequence. For \caafe, we instead select stars with $-0.15 \leq {\rm [Mg/H]} \leq -0.05$.  
To create a sample with similar characteristics of the subgiant observations, we sample 12,000 stars from the simulations such that the cumulative distribution function of stars in $R_{\rm gal}$ is the same for the subgiants and our comparison sample. 


As the primary observational constraint, we use the criteria outlined in \citetjack~to create a sample of subgiant from \apogee{} DR17 \citep{apogee17}. apogee is part of the Sloan Digital Sky Survey and measures high-resolution spectra of thousands of stars \cite{sdss17}. Chemical abundances are determined from the \apogee\ Stellar Parameter and Chemical Abundance Pipeline ({\sc aspcap}) \citep{aspcap}.  


Photospheric C and N abundances in subgiant are reflective of their birth abundances \citep{gilroy89, korn+07, lind+08, souto+18, souto19} As first dredge up, which affects C and N abundances, only occurs during the ascent onto the RGB, subgiant stars are unaffected by this enrichment. 

An alternate approach for this analysis would be to estimate the birth abundances of RGB stars by correcting surface abundance effects from first dredge up as in \cite{vincenzo+21}. Subgiants are the more attractive option since these observations do not rely on model-dependent corrections. However, RGB stars are more luminous, potentially allowing better coverage of the Galactic disk.


We choose to use \citetjack\ sample as this does not rely on additional layers of modeling, providing a more direct constraint to our model and limiting our systematic uncertainties.



Fig.~\ref{fig:subgiant_selection} shows a plot of all \apogee\ stars and the \citetjack polygon selection criteria. 
 \citetjack~select a region of stars based on surface gravity $\log g$, and effective surface temperature, $T_\text{eff}$.
 \begin{equation}
    \begin{cases} \label{eq:subgiant_selection}
        \log \text{g} \geq 3.5 \\
        \log \text{g} \leq 0.004\,T_{\rm eff} - 15.7 \\
        \log \text{g} \leq 0.000706\,T_{\rm eff} + 0.36 \\
        \log \text{g} \leq -0.0015\,T_{\rm eff} + 12.05 \\
        \log \text{g} \geq 0.0012\,T_{\rm eff} - 2.8. \\
    \end{cases}
\end{equation}
Additionally, we included stars in \apogee{} marked by the following flags.
\begin{description}
\item \verb|APOGEE_MIRCLUSTER_STAR|
\item \verb|APOGEE_EMISSION_STAR|
\item \verb|APOGEE_EMBEDDEDCLUSTER_STAR|
\item \verb|young cluster (IN-SYNC)|
\item \verb|APOGEE2_W345|
\item \verb|EB planet|
\end{description}
This cut isolates a clean sample of \about{12,000} subgiants.
We furthermore isolate the low- and high-$\alpha$ sequences with the cut
\begin{equation}\label{eq:high_alpha}
\begin{cases}
\text{[Mg/Fe]} >0.12-0.13\,\text{[Fe/H]}, & \text{[Fe/H]}<0\\
\text{[Mg/Fe]} >0.12, & \text{[Fe/H]}>0. \\
\end{cases}
\end{equation}
The low-$\alpha$ sequence is better reproduced by this model, so we use this cut of the subgiants to compare the models against except for comparing \caafe. 




\begin{figure}
    \centering
    \includegraphics{logg_jack.pdf}
    \caption[]{
        A Kiel diagram of \apogee{} stars. Following \citetjack, we select subgiants in the black pentagon (see Equation \ref{eq:subgiant_selection}). These stars have not yet experienced first dredge-up, so their photospheric C and N abundances should reflect their birth mixture.
    }
    \label{fig:subgiant_selection}
\end{figure}


\section{Results}

\subsection{Evolution of Carbon Abundances}

Here, we present the time evolution of our fiducial model. In the next sections, we will discuss the choice of parameters and agreement with observations. 
The fiducial model has the following qualitative characteristics of its C yields.
\begin{description}
    \item C is mostly (\about{80\%}) produced in \cc
    \item \cc\ produce more C at higher metallicities
    \item \agb\ stars produce less C at higher metallicities 
\end{description} 
The fiducial model uses the \cxi{} \agb\ yield tables uniformly scaled by a factor of 2.9 (see Section \ref{sec:agb}, and Table \ref{tab:fiducial_mod}). 

Fig.~\ref{fig:c_evo} shows time evolution tracks of the fiducial model for \caah\ and \caafe. 
As discussed in Section~\ref{sec:equilibrium}, \caah\ is set by the total C/Mg yields. 
\caafe\ is instead useful in understanding delayed C production. 
As both Fe and C are delayed elements, [Mg/Fe] steadily decreases after a star formation event, unlike [Mg/H] which quickly reaches equilibrium.  All plots showing \caafe\ going forward are selected in metallicity such that $-0.15 \leq {\rm [Mg/H]} \leq -0.05$, so metallicity-dependent yields do not affect this plot. 
The \caafe{}-diagram is, in essance, an emperical delay-time-distribution for a single stellar population of C, especially as we assume a $\propto t^{-1.1}$ delay-time-distribution for Fe. 
Comparing the left and right panels of Fig.~\ref{fig:c_evo} highlights the differences between \caah\ and \caafe. While \caah\ quickly reaches its final equilibrium distribution (within \about{5}\,Gyr), \caafe\ continues to evolve in both [Mg/Fe] and [C/Mg] until the simulation ends.

C evolution proceeds as follows, (for a single zone)
\begin{enumerate}
    \item \cc\ initially dominate production. As $\Ycc$ has strong metallicity dependence, [C/Mg] increases with time. 
    \item \agb\ stars contribute delayed C, causing [C/Mg] to increase even faster with [Mg/H]. 
    \item{} [C/Mg] plateaus as C also approaches equilibrium. 
    \item{} [C/Mg] may decrease due to declining SFH or slightly negative yields from \about{1}\,\Mo stars.

\end{enumerate}


\begin{figure*}
\centering
\includegraphics{all_the_tracks.pdf}
\caption[]{
    Time evolution of gas-phase C abundances in our fiducial model.
    Each line represents a zone at a different galactic radii. The lines are coloured-coded by time. The left shows \caah\ and the right \caafe. 
}
\label{fig:c_evo}
\end{figure*}




\subsection{High-Mass Stellar Yields}\label{sec:results_highmass}

\cc, which we assume make up the majority of C, set the overall C abundance trends.
The top of Fig.~\ref{fig:first_models} shows models with varying $\Ycc$ metallicity dependence, $\zeta^{\rm cc}$. As the \caah~trend is approximated by equilibrium trends, the models with higher $\zeta^{\rm cc}$ also have a steeper slope in \caah. 
However, \caafe~is minimally affected by changes to $\zeta^{\rm cc}$ since \cc\ occurs on much shorter timescales than \ia\ and \agb\ enrichment. The only effects on \caafe, when considering the narrow metallicity slice, are because of either the slight change in equilibrium abundances, the imperfect evolution of the galaxy, or that the interstellar medium abundances are set by stars which were born at poorer metallicities. 
 Hence, \caah\ tells us about the total C yield with metallicity, which \caafe\ is independent of. If we know the \agb\ C yields, then with observed \caafe\ abundance trends, we can infer the \cc\ C yields with metallicity.


\begin{figure*}
\includegraphics{all_p1.pdf}

\caption[]{
    Stellar abundance trends in our model, assuming metallicity independent $\Ycc$. Coloured lines represent the median [C/Mg] in bins of [Mg/H] (left) or [Mg/Fe] (right) for each model. Black points and grey dashes represent the median and standard deviations of [C/Mg] for each [Mg/H] bin in the \citetjack~sample. In the right panels, we show the trends only for stars where $-0.15\leq {\rm [Mg/H]}\leq -0.05$.
    \textbf{Top}: Models with different metallicity dependences for  \cc\ C yiels.
    \textbf{Middle}: Our four different \agb\ models.
    \textbf{Bottom}: Different \agb\ fractinos of C.
}
\label{fig:first_models}
\end{figure*}

\subsection{Low-Mass Stellar Yields}\label{sec:agb_results}


To parameterize the \agb\ contribution to C production, we define $f_{\rm agb}$ to be the fraction of C which comes from \agb\ stars. 
\begin{equation}\label{eq:f_agb}
    f_{\rm agb} \equiv \frac{\Ycagb(Z=\Zo)}{\Yct(Z=\Zo)},
\end{equation}
In Section \ref{sec:agb_results}, we will show that $f_{\rm agb} \approx 0.2$. 
None of the \agb\ yield sets (\cxi{}, \kx{}, \vxiii{}, \kxvi{}) produce enough C relative to our total $\Ycc$ values. So, we introduce a normalization factor $\alpha_{\rm agb}$, which denotes a multiplicative scaling of $\Ycagb$ 
\begin{equation} \label{eq:alpha}
        \Ycagb \rightarrow \alpha_{\rm agb}\ \Ycagb.
\end{equation}
Table~\ref{tab:alpha_agb} contains the required values of $\alpha_{\rm agb}$ to reach $f_{\rm agb}=0.2$, the yields at solar metallicity, and the metallicity for each \agb\ yield set. 

We can use the \caafe{} diagram of \apogee\ stars to estimate the delayed portion of C. When binned in metallicity, median [C/Mg] changes by about 0.2 dex across the range of [Mg/Fe] at solar metallicities. As high-$\alpha$ stars have little to no delayed \ia\ Fe, these stars would also have little to no delayed \agb\ C. This means that AGB C stars make up about at most a fraction $f_{\rm agb} \approx 1 - 10^{-0.2} \approx 0.4$ of C production.

\caafe\ is sensitive to the assumptions about delayed C from \agb\ stars. If no C comes from low-mass stars, then [C/Mg] would be independent of [Mg/Fe], only [Mg/H]. Instead, C shows strong trends  in \caafe\, independent of metallicity. 

In the middle row of Fig.~\ref{fig:first_models}, we first show the four C yield models (\kx, \kxvi, \cxi, \vxiii). For the most part, the \agb\ yield sets result in qualitatively similar predictions. \vxiii, however, does not reproduce solar trends as the model predicts strong C production at slightly above solar metallicities, resulting in a decreasing [C/Mg] with [Fe/Mg]. 
As all \agb\ models predict some low-mass C destruction, each model does predict a downturn in [C/Mg] as [Fe/Mg] increases. A recent burst in star formation may hide this downturn (see section~\ref{sec:sfh}), but the C destruction is not supported directly by observations. 



We also investigate adjustments to the \agb\ yield fraction $f_{\rm agb}$
The bottom of Fig.~\ref{fig:first_models} shows three models with different \agb\ fractions while using \cxi{} yields.  The \caafe~relationship is set by $f_{\rm agb}$ because a specific amount of C must be released at a delayed time to match the \ia\ production of Fe and increase [C/Mg] as [Mg/Fe] decreases to reproduce the data.
Increased $f_{\rm agb}$ results in a decreased slope in \caah, owing to the negative metallicity dependence of $\Ycagb$. So while \caah~alone cannot differentiate models which vary $f_{\rm agb}$ and $\zeta$ correspondingly, \caafe~provides information on $f_{\rm agb}$. So, we can use \caafe~to estimate $f_{\rm agb}\approx 0.2$, and then choose $\zeta$ to match \caah.


Finally, in the top row of Fig.~\ref{fig:second_models}, we plot FINISH where C production occurs in different mass ranges. If all C comes from stars with masses $>4\Mo$, then [C/Mg] is almost independent of [Fe/Mg]. On the other hand, if all C comes from stars with masses between 1 and 2 \Mo, then [C/Mg] increases across almost the whole range of [Fe/Mg]. As the observed trend in \caafe\ is almost linear, then the delay-time-distribution of C is similar to Fe, so C is likely produced in a mix of stars between 1 and 4\Mo. We cannot constrain upper-intermediate \agb\ C production as this behaves similar to \cc\ C.



\begin{figure*}
\centering
\includegraphics{nitrogen.pdf}

\caption[]{Similar to Fig.~\ref{fig:first_models}. The top panels show models with different mass-ranges contributing \agb\ C yields. The middle panels show different star formation histories (section \ref{sec:sfh}) and yield normalizations. The bottom panels show [N/Mg] against [Mg/H] and [Fe/Mg] for the fiducial model.
}
\label{fig:second_models}
\end{figure*}





\subsection{Star Formation History and Outflows} \label{sec:sfh}

In this section, we consider two modifications of our fiducial \sfh{}: \textit{lateburst} and \textit{earlyburst}.
Our lateburst model adds a Gaussian factor to the inside-out \sfh{},
\begin{equation}\label{eq:lateburst}
    \dot{\Sigma}_\text{lateburst} \propto \dot{\Sigma}_\text{inside-out} \left(1 + A\,e^{-(t-\tau_{\rm burst})^2/2\sigma^2_{\rm burst}} \right)
\end{equation}
where $A=1.5$ represents the amplitude of the birth, $\tau_\text{burst}=10.8$\,Gyr is the time where the burst is strongest, and $\sigma_\text{burst}=1$\,Gyr is the width of the burst.
% Our earlyburst model is instead a double exponential, with the second exponential begining at $t_1=5\,$Gyr.
% \begin{equation}\label{eq:twoinfall}
%     \dot{\Sigma}_{\rm earlyburst} \propto A\,e^{-t/\tau_{\rm burst}} + 
% \begin{cases}
%     e^{-(t-t_1)/\tau_\text{sfh}} & t_1 < t \\
%       0 & t<t_1
% \end{cases}
% \end{equation}
% where we take the burst duration, $\tau_{\rm burst}=2$\,Gyr.
% This approximately corresponds to the Gaia-Encelidus merger, inducing higher star formation in the Milky Way \citep{spitoni21, bonaca20, helmi18}.

The middle row of Fig.~\ref{fig:second_models} shows three models with our alternate \sfh{}. Changes to the \sfh{} leave \caah\ unchanged, but they do introduce slight variation in \caafe. Models with higher \agb{} fractions are more sensitive to variations in the \sfh{}. The late burst models result in [C/Mg] continuing to increase at low [Mg/Fe], but also introduce a dip not present in the data. Additionally, the early-burst
reproduces the slight break between the low and high $\alpha$ sequences, but overshoots equilibrium more severely than the fiducial model. 
In general, any of these \sfh{}s are consistent with this model.

The middle row of Fig.~\ref{fig:second_models} also shows a model where $\eta$ and yields are decreased by a factor 0.3. While changing the value of $\eta$ affects the metallicity distribution of stars, the model still evolves along the same paths. Our model is unable to differentiate a uniform decrease in both outflows and yields.


\subsection{Degeneracies} \label{sec:outflows}

Our conclutions are limited by the many uncertainties in \gce\ modeling. 

The overall scaling of yields and outflows is unknown and challenging to constrain.
\gce{} models of the Milky Way fall into two classes -- those which incorporate significant mass-loading (e.g., this work) and those which neglect mass-loading but lower effective yield to match observed abundances \citep[e.g.][]{MCM13, MCM14, spitoni19, spitoni20, spitoni21}.
An increase in stellar yields has a nearly identical effect as a decrease in the mass-loading factor $\eta$ (see Appendix B of \citealt{james_dwarf}).
The equilibrium arguments discussed in Section \ref{sec:equilibrium} suggest however that abundance ratios are independent of the choice of normalization and the value of $\eta$. We, therefore, expect our results regarding the relative yield $y_{\rm C}/y_{\rm Mg}$ and its metallicity dependence to extend to the other class of models omitting mass loading. We demonstrate this further here.
The theoretical motivation for decreasing yields is the uncertainty in stellar explodability.
If fewer \hms\ explode, then the yields will be reduced by some factor. Additionally, some fraction of supernovae ejecta may be lost directly to an outflow, lowering effective yields. To explore reduced outflow models, we lower both $\eta$ and all yields by the same factor to leave the equilibrium abundances unchanged. 


An addition source of theoretical uncertainty in this result is that the \ia\ yield and delay time distributions have their own uncertainties. We do compute models (not shown here) where $y_{\rm Fe}^{\rm Ia}$ is varied. Increasing $y_{\rm Fe}^{\rm Ia}$ has a similar affect as decreasing $f^{\rm agb}$. Here, we only show models with $y_{\rm Fe}^{\rm Ia} = 0.00214$ and a $t^{-1.1}$ delay-time-distribution choices from the fiducial model here.



\subsection{Gas-Phase Abundances}\label{sec:gas}

As a final test of the model, we compare the model predictions against gas-phase measurements. Fig.~\ref{fig:gas_phase} shows the fiducial model's gas-phase predictions compared to observations of the Milky Way and extragalactic HII regions, halo stars, and damped Lyman-alpha systems. 
While obervations in HII regions and Milky Way stars agree that C/O generally increases at near-solar metallicities, damped Lyman-alpha systems and metal poor stars imply that C/O may also increase again at very low metallicities. 



Measurements of C abundances are challenging. 
In HII regions, C/O abundance ratios are measured with either recombination lines or collisional excitation lines. While broad consistancy of our model with gas-phase measurments is promising, the large measurment errors limit the evaluative power of HII regions.
We additionally include Milky Way thick disk (high-$\alpha$) and halo stars, which span a larger range of metallicities than thin disk stars. However, metal-poor abundance measurments require consideration of 3D-NLTE effects, which can be an \about{0.2} dex effect  (e.g. CITE). 
Finally, dwarf galaxies, damped Lyman-alpha systems, and extragalactic regions may represent different \sfh{}s than the Milky Way.
As \agb\ C production is sensitive to variations in the \sfh, these environments may not exactly match our subgiant sample, limiting the usefulness of these abundance measurments in evaluating our model.

C lines are relatively faint, and surveys such as GALAH struggle against low detection rates, potentially biasing sample measurments. 
(e.g. gaia-eso AS WELL \citealt{franchini+20}).
In the gas phase, HII regions are our best window into C abundances. Unfortunantly, C lacks strong collisional excitation lines, and recombination lines fall in the ultraviolet without nearby reference H lines \citep{skillman+20}. Recombination-line and collisional-excitation-line measurements furthermore disagree by a factor of \about{2} \citep{GR07}. 



Fig.~\ref{fig:gas_phase} shows the single-zone model and time-slices of the fiducial multi-zone model at present day and $t=2$\,Gyr. 
Here, we consider a single-zone model with parameters consistant with the Gaia-Encelidus sausage\footnotemark{}. We chose the model to have mass loading $\eta=20$, star formation efficiency $\tau_{\star}=16\,{\rm Gyr}$, and a star formation history $\propto e^{-t/3{\rm\,Gyr}}$, evolved for 2\,Gyr \citep{james_dwarf}.
The single-zone model is better able to replicate the slope of the abundances in dwarf galaxies, HII-regions, and halo stars. The single-zone model does not produce an equilibrium track, unlike the multi-zone models. As the single-zone model also evolves slower, the late contribution of AGB stars causes the steeper slope at near-solar metallicities. By including an increase of C yields at low-metallicity, the single-zone model is also able to reproduce the increasing [C/O] abundances with decreasing metallicity past ${\rm [O/H]} < -1.5$. In any case, there is large scatter in the measurments, which both models fall within.



\footnotetext{See e.g. CITATION. Th Gaia-Encelidus sausage (GSE) is a kinematically and chemically distinct group of halo stars consistant with the merger of a dwarf galaxy early in the Milky Way's formation.}


\begin{figure*}
\centering
\includegraphics[]{summary.pdf}
\caption[]{Gas-phase C abundances. We plot our model at $t=2$\,Gyr and present day as thick solid lines. Black lines are single-zone models. Points represent measurements in 
    HII regions    \citep[pink circles;][]{skillman+20, esteban+02, esteban+09, esteban+14, esteban+19}
    damped Lyman-alpha (DLA) systems \citep[blue triangles;][]{ellison+10, srianand+10, dutta+14, DZ+03, pettini+08, morrison+16,cooke+17},  % a1: Cooke et al. (2015); 2: Dutta et al. (2014); 3: Cooke et al. (2014); 4: Ellison et al. (2010); 5: Cooke et al. (2011b); 6: This work; 7: Pettini et al. (2008); 8: Morrison et al. (2016); 9: Srianand et al. (2010); 10: Cooke et al. (2012); 11: Dessauges-Zavadsky et al. (2003)
    dwarf galaxies \citep[red diamonds;][]{berg+19},
    Milky Way halo and thick disk stars \citep[green stars;][]{amarsi+19, nissen+14, fabbian+09},
    and Milky Way high-$\alpha$ stars (yellow points; \citealtjack).
}
\label{fig:gas_phase}
\end{figure*}


\section{Conclusions}

In this work, we investigated the role of C yields on the predictions of multi-zone \gce{} models. 
We began by adopting an equilibrium approximation to estimate the total C yields with metallicity from \apogee{} subgiant \caah{} trends.
We find that $\Yct/y_{\rm O} = 1/3 + 4(Z-\Zo)$ and $\Yct/y_{\rm Mg} = 2.7 + 32 (Z-\Zo)$, where we assume ${\rm [Mg/O]} = 0$.
We show that \caah{} is a diagnostic for total C yields with metallicity, but \caafe{} provides information about delayed C production.
From the \caafe{} trends, we estimate that \agb\ stars with masses between about 1 and 3 \Mo contribute \about{20\%} of total C abundances. In this model, the remaining \about{80\%} of C comes from high-mass stars with a metallicity dependent yield of $\Ycc/y_{\rm Mg} = 2.2 + 55 (Z/\Zo)$, broadly consistant with rotating \cc\ models.

We additionally explore variations of the assumed \sfh{} and outflow mass-loading factor $\eta$. We find that alternate \sfh{}s can perturb \caafe and \caah\ abundances slightly. Decreasing both outflows and yields by the same factor leaves the \caah~and \caafe~trends unaffected, ingoring effect to the metalliticy distribution of starrs. These constraints on the relative yields of C, O, and Mg are robust against variations in $\eta$.

Finally, we compare our model against gas-phase measurements and Milky Way halo stars. By including yields which are enhanced at low metallicities, \cc\ and \agb\ stars together are able to explain the general trends of C from metallicities of ${\rm [O/H]} = -3$ to 0.5. 

Our C yield constraints provide a useful benchmark for stellar evolution models. C yields are sensitive to poorly understood processes, including mass-loss prescriptions, explodability, nuclear cross sections, convection, and stellar structure. Future spectroscopic surveys combined with Gaia kinematics \citep{gaia} will continue to enhance our understanding of chemical evolution. Both the Sloan Digital Sky Survey V's Milky Way Mapper program ({\sc SDSS-V/MWM}) \citep{sdssv} and the Dark Energy Spectroscopic Instrument ({\sc DESI}) Milky Way survey \citep{desi, desi:mw} will each measure spectra of upwards 6,000,000 Milky Way stars. These larger samples will enable similar work to tighten constraints on stellar models and our understanding of galaxy structure and evolution.



%%%%%%%%%%%%%%%%%%%%%%%%%%%%%%%%%%%%%%%%%%%%%Log in with your OSC username and password.

%%%%%
\section*{Acknowledgements}

 Here you can thank helpful
colleagues, acknowledge funding agencies, telescopes and facilities used etc.
Try to keep it short.

Software that has contributed to this work included  
\VICE~\citep{JW20, james+21},
\textsc{matplotlib} \citep{matplotlib},
\textsc{scipy} \citep{scipy},
\textsc{IPython} \citep{ipy},
\textsc{pandas} \citep{pandas},
\textsc{numpy} \citep{numpy},
\textsc{astropy} \citep{astropy:2013, astropy:2018, astropy:2022},
and 
\textsc{seaborn} \citep{seaborn}
.
Additionally, we thank \citet{OhioSupercomputerCenter1987} for the use of its facilities for the simulations. 


%%%%%%%%%%%%%%%%%%%%%%%%%%%%%%%%%%%%%%%%%%%%%%%%%%
\section*{Data Availability}

 
The inclusion of a Data Availability Statement is a requirement for articles published in MNRAS. Data Availability Statements provide a standardised format for readers to understand the availability of data underlying the research results described in the article. The statement may refer to original data generated in the course of the study or to third-party data analysed in the article. The statement should describe and provide means of access, where possible, by linking to the data or providing the required accession numbers for the relevant databases or DOIs.


%%%%%%%%%%%%%%%%%%%% REFERENCES %%%%%%%%%%%%%%%%%%
\bibliographystyle{mnras}
\bibliography{main}


%%%%%%%%%%%%%%%%% APPENDICES %%%%%%%%%%%%%%%%%%%%%

\appendix


\bsp	% typesetting comment
\label{lastpage}
\end{document}




